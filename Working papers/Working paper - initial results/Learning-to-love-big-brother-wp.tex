% Options for packages loaded elsewhere
\PassOptionsToPackage{unicode}{hyperref}
\PassOptionsToPackage{hyphens}{url}
\PassOptionsToPackage{dvipsnames,svgnames,x11names}{xcolor}
%
\documentclass[
  letterpaper,
  DIV=11,
  numbers=noendperiod]{scrartcl}

\usepackage{amsmath,amssymb}
\usepackage{iftex}
\ifPDFTeX
  \usepackage[T1]{fontenc}
  \usepackage[utf8]{inputenc}
  \usepackage{textcomp} % provide euro and other symbols
\else % if luatex or xetex
  \usepackage{unicode-math}
  \defaultfontfeatures{Scale=MatchLowercase}
  \defaultfontfeatures[\rmfamily]{Ligatures=TeX,Scale=1}
\fi
\usepackage{lmodern}
\ifPDFTeX\else  
    % xetex/luatex font selection
\fi
% Use upquote if available, for straight quotes in verbatim environments
\IfFileExists{upquote.sty}{\usepackage{upquote}}{}
\IfFileExists{microtype.sty}{% use microtype if available
  \usepackage[]{microtype}
  \UseMicrotypeSet[protrusion]{basicmath} % disable protrusion for tt fonts
}{}
\makeatletter
\@ifundefined{KOMAClassName}{% if non-KOMA class
  \IfFileExists{parskip.sty}{%
    \usepackage{parskip}
  }{% else
    \setlength{\parindent}{0pt}
    \setlength{\parskip}{6pt plus 2pt minus 1pt}}
}{% if KOMA class
  \KOMAoptions{parskip=half}}
\makeatother
\usepackage{xcolor}
\setlength{\emergencystretch}{3em} % prevent overfull lines
\setcounter{secnumdepth}{5}
% Make \paragraph and \subparagraph free-standing
\ifx\paragraph\undefined\else
  \let\oldparagraph\paragraph
  \renewcommand{\paragraph}[1]{\oldparagraph{#1}\mbox{}}
\fi
\ifx\subparagraph\undefined\else
  \let\oldsubparagraph\subparagraph
  \renewcommand{\subparagraph}[1]{\oldsubparagraph{#1}\mbox{}}
\fi


\providecommand{\tightlist}{%
  \setlength{\itemsep}{0pt}\setlength{\parskip}{0pt}}\usepackage{longtable,booktabs,array}
\usepackage{calc} % for calculating minipage widths
% Correct order of tables after \paragraph or \subparagraph
\usepackage{etoolbox}
\makeatletter
\patchcmd\longtable{\par}{\if@noskipsec\mbox{}\fi\par}{}{}
\makeatother
% Allow footnotes in longtable head/foot
\IfFileExists{footnotehyper.sty}{\usepackage{footnotehyper}}{\usepackage{footnote}}
\makesavenoteenv{longtable}
\usepackage{graphicx}
\makeatletter
\def\maxwidth{\ifdim\Gin@nat@width>\linewidth\linewidth\else\Gin@nat@width\fi}
\def\maxheight{\ifdim\Gin@nat@height>\textheight\textheight\else\Gin@nat@height\fi}
\makeatother
% Scale images if necessary, so that they will not overflow the page
% margins by default, and it is still possible to overwrite the defaults
% using explicit options in \includegraphics[width, height, ...]{}
\setkeys{Gin}{width=\maxwidth,height=\maxheight,keepaspectratio}
% Set default figure placement to htbp
\makeatletter
\def\fps@figure{htbp}
\makeatother
\newlength{\cslhangindent}
\setlength{\cslhangindent}{1.5em}
\newlength{\csllabelwidth}
\setlength{\csllabelwidth}{3em}
\newlength{\cslentryspacingunit} % times entry-spacing
\setlength{\cslentryspacingunit}{\parskip}
\newenvironment{CSLReferences}[2] % #1 hanging-ident, #2 entry spacing
 {% don't indent paragraphs
  \setlength{\parindent}{0pt}
  % turn on hanging indent if param 1 is 1
  \ifodd #1
  \let\oldpar\par
  \def\par{\hangindent=\cslhangindent\oldpar}
  \fi
  % set entry spacing
  \setlength{\parskip}{#2\cslentryspacingunit}
 }%
 {}
\usepackage{calc}
\newcommand{\CSLBlock}[1]{#1\hfill\break}
\newcommand{\CSLLeftMargin}[1]{\parbox[t]{\csllabelwidth}{#1}}
\newcommand{\CSLRightInline}[1]{\parbox[t]{\linewidth - \csllabelwidth}{#1}\break}
\newcommand{\CSLIndent}[1]{\hspace{\cslhangindent}#1}

\usepackage{booktabs}
\usepackage{longtable}
\usepackage{array}
\usepackage{multirow}
\usepackage{wrapfig}
\usepackage{float}
\usepackage{colortbl}
\usepackage{pdflscape}
\usepackage{tabu}
\usepackage{threeparttable}
\usepackage{threeparttablex}
\usepackage[normalem]{ulem}
\usepackage{makecell}
\usepackage{xcolor}
\usepackage{siunitx}

  \newcolumntype{d}{S[
    input-open-uncertainty=,
    input-close-uncertainty=,
    parse-numbers = false,
    table-align-text-pre=false,
    table-align-text-post=false
  ]}
  
\KOMAoption{captions}{tableheading}
\makeatletter
\makeatother
\makeatletter
\makeatother
\makeatletter
\@ifpackageloaded{caption}{}{\usepackage{caption}}
\AtBeginDocument{%
\ifdefined\contentsname
  \renewcommand*\contentsname{Table of contents}
\else
  \newcommand\contentsname{Table of contents}
\fi
\ifdefined\listfigurename
  \renewcommand*\listfigurename{List of Figures}
\else
  \newcommand\listfigurename{List of Figures}
\fi
\ifdefined\listtablename
  \renewcommand*\listtablename{List of Tables}
\else
  \newcommand\listtablename{List of Tables}
\fi
\ifdefined\figurename
  \renewcommand*\figurename{Figure}
\else
  \newcommand\figurename{Figure}
\fi
\ifdefined\tablename
  \renewcommand*\tablename{Table}
\else
  \newcommand\tablename{Table}
\fi
}
\@ifpackageloaded{float}{}{\usepackage{float}}
\floatstyle{ruled}
\@ifundefined{c@chapter}{\newfloat{codelisting}{h}{lop}}{\newfloat{codelisting}{h}{lop}[chapter]}
\floatname{codelisting}{Listing}
\newcommand*\listoflistings{\listof{codelisting}{List of Listings}}
\makeatother
\makeatletter
\@ifpackageloaded{caption}{}{\usepackage{caption}}
\@ifpackageloaded{subcaption}{}{\usepackage{subcaption}}
\makeatother
\makeatletter
\@ifpackageloaded{tcolorbox}{}{\usepackage[skins,breakable]{tcolorbox}}
\makeatother
\makeatletter
\@ifundefined{shadecolor}{\definecolor{shadecolor}{rgb}{.97, .97, .97}}
\makeatother
\makeatletter
\makeatother
\makeatletter
\makeatother
\ifLuaTeX
  \usepackage{selnolig}  % disable illegal ligatures
\fi
\IfFileExists{bookmark.sty}{\usepackage{bookmark}}{\usepackage{hyperref}}
\IfFileExists{xurl.sty}{\usepackage{xurl}}{} % add URL line breaks if available
\urlstyle{same} % disable monospaced font for URLs
\hypersetup{
  pdftitle={Learning to love big brother: Chinese attitudes toward online privacy after the pandemic},
  pdfauthor={Andrew MacDonald},
  pdfkeywords={Digital Privacy, Covid-19, Authoriatrian
Regimes, Government Trust, China},
  colorlinks=true,
  linkcolor={blue},
  filecolor={Maroon},
  citecolor={Blue},
  urlcolor={Blue},
  pdfcreator={LaTeX via pandoc}}

\title{Learning to love big brother: Chinese attitudes toward online
privacy after the pandemic}
\author{Andrew MacDonald}
\date{2024-01-11}

\begin{document}
\maketitle
\begin{abstract}
The results of the survey strongly suggest that, despite all of the
invasive government monitoring used to attempt to control the spread of
Covid, respondents were generally supportive of government invasions of
their privacy. These findings cast doubt on the long-term impact of the
White Paper movement protests and connect to a larger body of literature
on why surveys of Chinese citizens indicate high levels of trust in
their government.
\end{abstract}
\ifdefined\Shaded\renewenvironment{Shaded}{\begin{tcolorbox}[breakable, sharp corners, boxrule=0pt, interior hidden, borderline west={3pt}{0pt}{shadecolor}, enhanced, frame hidden]}{\end{tcolorbox}}\fi

\hypertarget{introduction}{%
\section{Introduction}\label{introduction}}

The trauamtic period of the Covid-19 controls in China in 2022 lead many
news commentators to believe that the Chinese regime was in deep
trouble. In particular, alarm was raised in Western news and scholarly
sources regarding China's increasingly (as the sources term it)
dystopian digital surveillance state (for some examples of these
reports, see (Ang, 2022; Intercepted, 2022; Vlahos, 2019)). However,
existing literature also finds a deep wellspring of trust among Chinese
citizens toward the state and general lack of concern regarding
government surveillance of their online activities (Chen, 2017;
Steinhardt et al., 2022). This report seeks to examine whether the
events of 2022 in China significantly altered 1) citizen's attitudes
about government monitoring and 2) the overall trust in government. It
does so by analyzing the results of two waves of a survey focused on
Chinese citizens' attitudes toward online privacy, with one wave
conducted in the spring of 2021 and the other in the spring of 2023. The
results suggest that Covid-19 controls, if anything, increased
respondent support for government digital monitoring and only modestly
decreased trust in the state. The one major exception was the
respondents in the city of Shanghai, where there was a large and
significant decrease in both measures between the two waves. Taken
together, these findings suggest that the type, duration, and severity
of monitoring is an important variable in public acceptance of
monitoring.

\hypertarget{sec-litreview}{%
\section{Literature review}\label{sec-litreview}}

The Covid-19 restrictions in China in 2022 were perhaps one of the most
extreme forms of digital control ever enacted. Residents were required
to show a digital health code to enter public spaces (Dou, 2022;
McMorrow and Leng, 2022). A separate app generated a travel pass code
based on which cell phone towers a resident's mobile phone had connected
to. City residents were required to upload regular Covid-19 tests or
face bans from entering most public areas, including city parks. One
relatively unique aspect of the technological controls was its
public-private nature. The health code end user interface existed within
the two most popular consumer app platforms in China, Alipay and WeChat.
These interfaces connected with provincial or local government-run
databases to log user scans and to generate current health code status.
In China, these public-private technological monitoring solutions are
relatively common yet how citizens react to and receive these solutions
remain understudied and under-theorized (Steinhardt et al., 2022).

The overall Covid-19 control regime in China in 2022 generated
significant resistance among the population. The most iconic episode of
the Chinese government's attempt to control the virus was the 60 day
Shanghai lockdown. Residents fumed online at being shut indoors, being
denied access to vital medicines or access to medical facilities, and
needing to scrounge for food (Dou et al., 2022; Lin and Jie, 2022). By
November, many Chinese citizens had enough of the controls. After a
number of fatalities arising from a fire in a building with all the
emergency doors padlocked to prevent quarantining residents from
escaping, young people reacted by launching the so-called White Paper
movement. Youth around China held up blank sheets of white paper,
protesting their lack of free speech to comment on the Covid-19
controls. Shortly thereafter, according to reporting from Western news
sources, Xi Jinping acceded to the demands of the protesters and the
Covid-19 restrictions were finally relaxed in the winter of 2022 (Huang
and Han, 2022).

From a theoretical perspective, one may expect that the impact of such
draconian forms of control would shift public views against government
monitoring and surveillance, especially in an authoritarian context.
Authoritarian leaders need to solve the information dilemma presented by
preference falsification - citizens are disincentivized from revealing
their true thoughts on policies or the regime due to fear of punishment
(Xu, 2021). Therefore, many authoritarian regimes employ a large
surveillance apparatus to monitor private spaces in an attempt to detect
the true feelings of residents. The classic response of citizens in the
countries behind the former Iron Curtain to such controls was to
withdraw inward into even more private spaces (PFAFF, 2001). In the long
run, this heavy-handed surveillance actually worked against regime goals
by decreasing regime support. One aspect of Chinese digital surveillance
has been that, for most citizens, it feels like a light touch. In
previous surveys, many Chinese respondents do not report feeling
excessively monitored or censored in any way (Gainous et al.,
2023).\footnote{This response may be due to the way digital censorship
  and monitoring is implemented in China. Since it is a public-private
  partnership, usually the first instance of control is by the platform;
  the hand of the state is not visible unless citizens engage in
  something more than simply blowing off steam (King et al., 2013).}
However, if theoretical expectation is true for the case of China, one
would expect that moving from a regime of light touch surveillance to
one more similar to that used in former Soviet-bloc countries would
decrease support for both surveillance and the regime itself.

\(H_1\): acceptance of surveillance and regime support among Chinese
survey respondents decreased after the digital Covid-19 controls of
2022.

Robin Li, founder of Baidu, expresses another common expectation of
Chinese citizen with regard to their attitude toward privacy, in which
he claimed that, ``I think that the Chinese people are more open, or not
so sensitive, about the privacy issue. If they are able to exchange
privacy for convenience or efficiency, they are willing to do so in many
cases'' \emph{People's Daily Online} (2018). Li ignited significant
controversy on the Chinese internet with his claims, with most users
criticizing or mocking his claims {[}Shen (2018){]}. More recent
scholarly research has refined this his claim, with findings that
generally suggest users have a significantly greater concern for their
privacy when considering private corporations compared to government
entities (Kostka, 2019; Steinhardt et al., 2022; Wang and Yu, 2015;
Zhang et al., 2002).

\(H_2:\) concern for user's personal privacy will be greater with
respect to corporations compared to government entities

If we combine \(H_1\) with \(H_2\), it would follow that another
expectation should be that concern about government monitoring should
increase, though perhaps from a low level, as citizens experience more
coercive forms of digital control.

\(H_3:\) concern for government monitoring should start at a low level
but increase after the events of 2022

In a separate line of literature, recent research on privacy in
democratic socieities has suggested that acceptance of digital
surveillance is tied to the level of government trust. Respondents with
high levels of trust in the government are more willing to accept
government monitoring (Ioannou and Tussyadiah, 2021; Reddick et al.,
2015; Trüdinger and Steckermeier, 2017). These theories may help explain
why previous studies have found a high degree of acceptance of
government monitoring, as Chinese respondents, in many different
contexts and studies, have repeatedly indicated that they have a high
level of trust in the Central government (Kennedy, 2009; Zhong, 2014),
though with lesser levels of trust in local governments. Therefore, the
decisive factor in changing attitudes toward online monitoring may be a
decrease in government trust during the pandemic rather than specific
views about technological controls.

\(H_4:\) views regarding online monitoring should move in line with
changing views of government trust in 2022

In the next section, I explore a two-wave survey to examine each of
these hypotheses in turn.

\hypertarget{sec-datasummary}{%
\section{Data and summary statistics}\label{sec-datasummary}}

The data for this project was collected via a commercial survey firm in
two waves, February of 2021 and March of 2023. In both the first and
second waves, Wuhan was oversampled, with residents of the city set to
be 10\% of respondents. The 2021 survey had an n=1500 and the second had
an n=2000. Questions on the two surveys were identical other than a
minor change to a question that referenced a specific date. The timing
of the two surveys came at two very different points in time of China's
Covid-19 experience. The first survey was conducted approximately seven
months after the last round of restrictions were lifted on the city of
Wuhan. China, at the time, was essentially closed to foreign travel but
otherwise had little in the way of day to day public health
restrictions. Nationwide, daily Covid cases hovered around the single
digits (\emph{BBC News}, 2021). China was at a very different point in
its journey in March of 2023. The year of 2022 saw widespread, intrusive
digital monitoring introduced. Many major cities, such as Shanghai,
Xi'an, and Shenzhen, underwent long and painful city-wide lockdown
procedures. At the end of 2023, under the weight of a spiraling number
of cases and widespread protests (termed the White Paper Revolution),
China finally abandoned its zero Covid policy (Mao, 2022). The two waves
of these surveys aim to compare attitudes before and after this
widespread and highly visible change in digital monitoring strategies.

The demographics of the 2021 and 2023 surveys are presented in
Table~\ref{tbl-demographics}.

\hypertarget{tbl-demographics}{}
\begin{longtable}[t]{llrr}
\caption{\label{tbl-demographics}Select key demographic variables }\tabularnewline

\toprule
  &    & Mean & Std. Dev.\\
\midrule
Age &  & 33.2 & 11.6\\
\midrule
 &  & N & Pct.\\
Location & Countryside/village & 477 & 13.6\\
 & Small city & 1059 & 30.2\\
 & Mid-sized city & 840 & 24.0\\
 & Big city & 1131 & 32.2\\
Education & No formal education & 22 & 0.6\\
 & Primary & 134 & 3.8\\
 & Middle school & 384 & 10.9\\
 & High school & 843 & 24.0\\
 & University & 1914 & 54.6\\
 & Advanced studies/Graduate school & 210 & 6.0\\
Gender & Female & 1711 & 48.8\\
 & Male & 1796 & 51.2\\
Marriage status & Single & 1101 & 31.4\\
 & In a relationship & 569 & 16.2\\
 & Married & 1744 & 49.7\\
 & Divorced & 93 & 2.7\\
Party member status & Yes & 483 & 13.8\\
 & No & 3024 & 86.2\\
Communist Youth League status & Yes & 1116 & 31.8\\
 & No & 2391 & 68.2\\
Income & 0-2,999 & 275 & 7.8\\
 & 3,000-5,999 & 822 & 23.4\\
 & 6,000-9,999 & 899 & 25.6\\
 & 10,000-19,999 & 962 & 27.4\\
 & 20,000-49,999 & 385 & 11.0\\
 & 50,000-99,999 & 94 & 2.7\\
 & More than 100,000 & 70 & 2.0\\
Year & 2021 & 1500 & 42.8\\
 & 2023 & 2007 & 57.2\\
\bottomrule
\end{longtable}

As is typical of online surveys in China, the sample respondents skew
somewhat younger and more educated. Comparing the two waves, there are
some modest demographic differences (notably education and marriage)
differences between the two samples. As will be shown in
Section~\ref{sec-analysis}, these minor differences do not appear to
change any of the substantive results. Focusing on the 2023 survey, the
modal respondent is someone from a small city, male, married, working in
a white collar job at a small enterprise, who earns about 10,000 RMB a
month and has an urban \emph{hukou}. This demographic profile already
suggests that while the large-scale lockdowns that occurred in a few of
the big cities generated a lot of press, they may not be the modal or
average citizen's experience with zero Covid policies.

Taking a wide-angle view on the government's performance,
Table~\ref{tbl-gp.summary.data} compares some different measures of
government performance. While the higher level of government trust in
the central government is unsurprising and consistent with previous
literature, the magnitude of the gap is somewhat smaller than in
previous studies. There has been a small but statistically significant
decrease in trust of both since 2021. Most interestingly, though, is
that there was no decrease in how residents feel the government handled
their privacy information. This result is some evidence against
hypothesis \(H_4\), that the two measures should move together. It also
calls into question \(H_1\) and \(H_2\), as the post-Covid controls wave
does not appear to have increased concern for government monitoring.

\hypertarget{tbl-gp.q.text}{}
\begin{table}
\caption{\label{tbl-gp.q.text}Government performance questions }\tabularnewline

\centering
\begin{tabular}[t]{l|>{\raggedright\arraybackslash}p{5in}}
\hline
Q1 & Overall, I’m happy with the performance of the central government\\
\hline
\cellcolor{gray!6}{Q2} & \cellcolor{gray!6}{Overall, I’m happy with the performance of my local government}\\
\hline
Q3 & The government does a good job balancing the rights of citizens to be free of surveillance in their daily lives with the need to preserve order and prevent crime.\\
\hline
\cellcolor{gray!6}{Q4} & \cellcolor{gray!6}{Government performance index of Q1 + Q2 + Q3}\\
\hline
\end{tabular}
\end{table}

\hypertarget{tbl-gp.summary.data}{}
\begin{table}
\caption{\label{tbl-gp.summary.data}Government performance data }\tabularnewline

\centering
\resizebox{\linewidth}{!}{
\begin{tabular}[t]{lrrrrrr}
\toprule
\multicolumn{1}{c}{ } & \multicolumn{2}{c}{2021 (N=1500)} & \multicolumn{2}{c}{2023 (N=2007)} & \multicolumn{2}{c}{ } \\
\cmidrule(l{3pt}r{3pt}){2-3} \cmidrule(l{3pt}r{3pt}){4-5}
  & Mean & Std. Dev. & Mean & Std. Dev. & Diff. in Means & p\\
\midrule
\cellcolor{gray!6}{Central government performance} & \cellcolor{gray!6}{5.98} & \cellcolor{gray!6}{1.14} & \cellcolor{gray!6}{5.73} & \cellcolor{gray!6}{1.36} & \cellcolor{gray!6}{-0.25} & \cellcolor{gray!6}{0.00}\\
Local government performance & 5.55 & 1.25 & 5.35 & 1.43 & -0.20 & 0.00\\
\cellcolor{gray!6}{Government performance Q3} & \cellcolor{gray!6}{5.45} & \cellcolor{gray!6}{1.22} & \cellcolor{gray!6}{5.42} & \cellcolor{gray!6}{1.41} & \cellcolor{gray!6}{-0.03} & \cellcolor{gray!6}{0.44}\\
Government performance index & 0.78 & 0.17 & 0.75 & 0.21 & -0.03 & 0.00\\
\bottomrule
\end{tabular}}
\end{table}

Turning to the issue of specific attitudes regarding monitoring,
Table~\ref{tbl-gm.summary.data} suggests that while the differences are
not large, respondents in the second wave were more accepting of almost
all forms of monitoring. Given the phrasing of the question - ``there
are good reasons for the government to monitor you'', it seems likely
that respondents were accepting the government's framing that such
digital monitoring and control was a necessary part of the pandemic
response. Not surprisingly, and consistent with previous research,
respondents trust the government at significantly higher levels than
private corporations. While the pandemic-era monitoring was in fact a
public-private partnership, respondents seem to have a very clear
delineation of which parties may acceptably gather their data and which
should not. These results are in line with \(H_2\), that respondents
should have greater trust in the government to monitor them online
compared to private corporations.

The change in these variables from 2021 to 2023 is one of the largest
differences among all survey questions. In 2023, respondents felt that
the central government had a significantly stronger reason to monitor
them compared to 2021. Respondents also agreed that the local government
had a better case to monitor them compared to 2021, though the magnitude
of the change was not as dramatic. Other types of monitoring (private
monitoring, biometric monitoring) also exhibited a statistically
significant change in the direction of being more accepting of tracking.
The most direct interpretation of this response pattern is that
respondents fundamentally accepted the government's position that
monitoring was necessary and justified during the pandemic. Contrary to
Western popular press reports of the White Paper Revolution, it does not
seem that most Chinese citizens were unhappy about Covid electronic
controls due to the feeling of being monitored. The slight decrease in
government trust could indicate dissatisfaction with other Covid
policies (including forced quarantines), but it does not seem that
app-based monitoring itself caused any great concern among the general
public. These results suggest that the motivation and type of monitoring
are also important factors in public acceptance of online monitoring,
factors not significantly considered in previous literature.

\hypertarget{tbl-gm.q.text}{}
\begin{table}
\caption{\label{tbl-gm.q.text}Government and private monitoring questions }\tabularnewline

\centering
\begin{tabular}[t]{l|>{\raggedright\arraybackslash}p{5in}}
\hline
Q1 & There are good reasons for the central government to monitor the activity of users online\\
\hline
\cellcolor{gray!6}{Q2} & \cellcolor{gray!6}{There are good reasons for the local government to monitor the activity of users online}\\
\hline
Q3 & There are good reasons for private companies to monitor the activity of users online\\
\hline
\cellcolor{gray!6}{Q4} & \cellcolor{gray!6}{It doesn’t bother me to provide the government with biometric information including my fingerprints and face details for the purposes of monitoring public places}\\
\hline
Q5 & It doesn’t bother me to provide private companies with biometric information including my fingerprints and face details for the purposes of monitoring public places\\
\hline
\cellcolor{gray!6}{Q6} & \cellcolor{gray!6}{Government monitoring index of Q1 + Q2 + Q4}\\
\hline
Q7 & Private monitoring index of Q3 + Q5\\
\hline
\cellcolor{gray!6}{Q8} & \cellcolor{gray!6}{Total monitoring index of Q1-Q5}\\
\hline
\end{tabular}
\end{table}

\hypertarget{tbl-gm.summary.data}{}
\begin{table}
\caption{\label{tbl-gm.summary.data}Government and private monitoring data }\tabularnewline

\centering
\resizebox{\linewidth}{!}{
\begin{tabular}[t]{lrrrrrr}
\toprule
\multicolumn{1}{c}{ } & \multicolumn{2}{c}{2021 (N=1500)} & \multicolumn{2}{c}{2023 (N=2007)} & \multicolumn{2}{c}{ } \\
\cmidrule(l{3pt}r{3pt}){2-3} \cmidrule(l{3pt}r{3pt}){4-5}
  & Mean & Std. Dev. & Mean & Std. Dev. & Diff. in Means & p\\
\midrule
\cellcolor{gray!6}{Central government monitoring} & \cellcolor{gray!6}{4.67} & \cellcolor{gray!6}{1.46} & \cellcolor{gray!6}{5.25} & \cellcolor{gray!6}{1.39} & \cellcolor{gray!6}{0.59} & \cellcolor{gray!6}{0.00}\\
Local government monitoring & 4.66 & 1.37 & 5.04 & 1.46 & 0.38 & 0.00\\
\cellcolor{gray!6}{Private company monitoring} & \cellcolor{gray!6}{2.91} & \cellcolor{gray!6}{1.56} & \cellcolor{gray!6}{3.08} & \cellcolor{gray!6}{1.84} & \cellcolor{gray!6}{0.18} & \cellcolor{gray!6}{0.00}\\
Government biometric monitoring & 4.80 & 1.50 & 5.00 & 1.59 & 0.21 & 0.00\\
\cellcolor{gray!6}{Private biometric monitoring} & \cellcolor{gray!6}{2.87} & \cellcolor{gray!6}{1.60} & \cellcolor{gray!6}{2.88} & \cellcolor{gray!6}{1.85} & \cellcolor{gray!6}{0.01} & \cellcolor{gray!6}{0.88}\\
Government monitoring index & 0.62 & 0.21 & 0.68 & 0.22 & 0.07 & 0.00\\
\cellcolor{gray!6}{Private monitoring index} & \cellcolor{gray!6}{0.31} & \cellcolor{gray!6}{0.23} & \cellcolor{gray!6}{0.33} & \cellcolor{gray!6}{0.29} & \cellcolor{gray!6}{0.02} & \cellcolor{gray!6}{0.08}\\
Total monitoring index & 0.50 & 0.17 & 0.54 & 0.19 & 0.05 & 0.00\\
\bottomrule
\end{tabular}}
\end{table}

One possible reason why respondents may believe that private
corporations are less trustworthy could arise from feeling that their
monitoring is more invasive. However, this turns out not to be the case
- respondents feel that all three entities are roughly equally likely to
monitor them. A likely interpretation of this result is that respondents
are unable to precisely identify who is monitoring them and when. When
the question about privacy is rephrased to further emphasize that these
different groups could access their private information, trust with
government sources decreases modestly compared to the previous phrasing
in Table~\ref{tbl-gm.q.text}. However, this decrease is matched by a
similar decrease in comfort with private companies monitoring them.
These results reinforce the results in Table~\ref{tbl-gm.summary.data}
and further support \(H_2\).

What may explain the lack of increased concern about digital privacy is
that respondents generally did not notice a major change in the level of
online monitoring. While the increase was statistically significant, it
was just barely at the edge of significance and amounts to less than
1/10th of a standard deviation increase in perceived monitoring. Given
the invasiveness of the technological means of control employed to
manage Covid, this result is surprising. Two reasonable explanations for
this divergence are 1) respondents do not consider the Covid controls to
be online monitoring and/or 2) respondents have already readjusted their
frame of reference and no longer bring to mind the Covid-19 era when
answering this question. While the survey results cannot arbitrate
between these two explanations, both of these explanations suggest that
even a massive and intrusive increase in surveillance has not made a
significant impact on how intrusive people view government monitoring.
If the first explanation is true, it suggests that the kinds of
monitoring that Chinese citizens are worried about are drastically
different than the kinds Western privacy advocates are concerned about.
If a government is monitoring every location you visit and such activity
is not considered to be particularly intrusive, then it suggests a very
different set of ideas about what is a concerning type of monitoring. If
the second explanation is true, it does suggest that respondents view
somewhat time-limited surveillance and surveillance for a specific
purpose as being acceptable.

\hypertarget{tbl-track.q.text}{}
\begin{table}
\caption{\label{tbl-track.q.text}Attitudes regarding tracking questions }\tabularnewline

\centering
\begin{tabular}[t]{l|>{\raggedright\arraybackslash}p{5in}}
\hline
Q1 & How closely do you think the central government tracks your online activity?\\
\hline
\cellcolor{gray!6}{Q2} & \cellcolor{gray!6}{How closely do you think the local government tracks your online activity?}\\
\hline
Q3 & How closely do you think private companies track your online activity?\\
\hline
\cellcolor{gray!6}{Q4} & \cellcolor{gray!6}{How comfortable are you with the central government knowing personal details about your activity online?}\\
\hline
Q5 & How comfortable are you with the local government knowing personal details about your activity online?\\
\hline
\cellcolor{gray!6}{Q6} & \cellcolor{gray!6}{How comfortable are you with private companies knowing personal details about your activity online?}\\
\hline
\end{tabular}
\end{table}

\hypertarget{tbl-track.summary.data}{}
\begin{table}
\caption{\label{tbl-track.summary.data}Attitudes regarding tracking summary data }\tabularnewline

\centering
\resizebox{\linewidth}{!}{
\begin{tabular}[t]{lrrrrrr}
\toprule
\multicolumn{1}{c}{ } & \multicolumn{2}{c}{2021 (N=1500)} & \multicolumn{2}{c}{2023 (N=2007)} & \multicolumn{2}{c}{ } \\
\cmidrule(l{3pt}r{3pt}){2-3} \cmidrule(l{3pt}r{3pt}){4-5}
  & Mean & Std. Dev. & Mean & Std. Dev. & Diff. in Means & p\\
\midrule
\cellcolor{gray!6}{Central government tracking - prevalence} & \cellcolor{gray!6}{4.31} & \cellcolor{gray!6}{1.30} & \cellcolor{gray!6}{4.40} & \cellcolor{gray!6}{1.44} & \cellcolor{gray!6}{0.09} & \cellcolor{gray!6}{0.05}\\
Local government tracking - prevalence & 4.22 & 1.28 & 4.35 & 1.45 & 0.13 & 0.01\\
\cellcolor{gray!6}{Private company tracking - prevalence} & \cellcolor{gray!6}{4.37} & \cellcolor{gray!6}{1.56} & \cellcolor{gray!6}{4.32} & \cellcolor{gray!6}{1.71} & \cellcolor{gray!6}{-0.05} & \cellcolor{gray!6}{0.34}\\
Central government tracking - comfort & 4.22 & 1.48 & 4.13 & 1.61 & -0.09 & 0.10\\
\cellcolor{gray!6}{Local govenrment tracking - comfort} & \cellcolor{gray!6}{4.09} & \cellcolor{gray!6}{1.49} & \cellcolor{gray!6}{4.05} & \cellcolor{gray!6}{1.63} & \cellcolor{gray!6}{-0.04} & \cellcolor{gray!6}{0.44}\\
Private company tracking - comfort & 2.64 & 1.68 & 2.54 & 1.78 & -0.11 & 0.07\\
\bottomrule
\end{tabular}}
\end{table}

Rounding out the final section of the regular survey questions are a set
of questions designed to further parse attitudes about online
monitoring, the results of which are shown in
Table~\ref{tbl-misc.summary.data}. The results of these questions
generally confirm and support the findings of all the previous question
blocks. As with most of the survey questions, there appeared to be only
a very modest change in response patterns between 2021 and 2023.
Respondents generally agree that they do not notice government tracking.
They strongly agree that the government protects their data better than
private corporations. They strongly disagree that they are willing to
give up their privacy simply to use apps for free. Finally, many
respondents feel worried about having their payment data stolen. While
it is unwise to place too much emphasis on any one question, given
respondents can misinterpret or gloss over any specific question,
respondents have repeatedly indicated that they trust the government at
significantly higher rates than private corporations regarding online
montoring and that most of users worries about being online are related
to protection of their information from corporations.

\hypertarget{tbl-misc.q.text}{}
\begin{table}
\caption{\label{tbl-misc.q.text}Attitudes on general questions }\tabularnewline

\centering
\begin{tabular}[t]{l|>{\raggedright\arraybackslash}p{5in}}
\hline
Q1 & I don’t notice government use of technology to monitor my behavior in my daily life.\\
\hline
\cellcolor{gray!6}{Q2} & \cellcolor{gray!6}{The government is likely to securely store my online personal data and information better than private companies.}\\
\hline
Q3 & It doesn’t bother me if private companies sell my user data to third parties if it will allow me to use their applications for free.\\
\hline
\cellcolor{gray!6}{Q4} & \cellcolor{gray!6}{I’m worried that my payment information might be stolen or compromised.}\\
\hline
\end{tabular}
\end{table}

\hypertarget{tbl-misc.summary.data}{}
\begin{table}
\caption{\label{tbl-misc.summary.data}General questions data }\tabularnewline

\centering
\resizebox{\linewidth}{!}{
\begin{tabular}[t]{lrrrrrr}
\toprule
\multicolumn{1}{c}{ } & \multicolumn{2}{c}{2021 (N=1500)} & \multicolumn{2}{c}{2023 (N=2007)} & \multicolumn{2}{c}{ } \\
\cmidrule(l{3pt}r{3pt}){2-3} \cmidrule(l{3pt}r{3pt}){4-5}
  & Mean & Std. Dev. & Mean & Std. Dev. & Diff. in Means & p\\
\midrule
\cellcolor{gray!6}{Do not notice government tracking} & \cellcolor{gray!6}{4.27} & \cellcolor{gray!6}{1.34} & \cellcolor{gray!6}{4.46} & \cellcolor{gray!6}{1.51} & \cellcolor{gray!6}{0.20} & \cellcolor{gray!6}{0.00}\\
Government secures data better than private & 5.50 & 1.27 & 5.48 & 1.36 & -0.02 & 0.71\\
\cellcolor{gray!6}{OK if apps sell my data so can use for free} & \cellcolor{gray!6}{2.27} & \cellcolor{gray!6}{1.55} & \cellcolor{gray!6}{2.44} & \cellcolor{gray!6}{1.77} & \cellcolor{gray!6}{0.17} & \cellcolor{gray!6}{0.00}\\
Payment data stolen worries & 5.64 & 1.27 & 5.50 & 1.53 & -0.14 & 0.00\\
\bottomrule
\end{tabular}}
\end{table}

One obvious objection to the finding that respondents have a higher
degree of concern with private monitoring compared to government
monitoring is that respondents are engaging in preference falsification
- they may be worried about, either consciously or subconsciously,
marking the government negatively in a survey. To address this concern,
the end of the survey employed a list experiment to measure variation in
levels of trust. The list experiment question gives respondents a list
of organizations that they may trust and then asks them to report the
number of organizations that they trust. Half of the respondents were
given a list of organizations that included a sensitive organization
(such as the central government). The other half was given a list
without the sensitive organization included. The idea is that
respondents may be more comfortable reporting that they do not trust an
organization when they do not have to consciously mark on a survey that
they do not trust it but instead is part of a mental math calculation
along with other items (Blair and Imai, 2012). List experiments have
been used across many fields to study sensitive topics such as racism,
abortion, and sexual violence (Moseson et al., 2017; Redlawsk et al.,
2010; Traunmüller et al., 2019). Since trust in corporations does not
seem likely to generate preference falsification problems, they were not
included as a separate list experiment. However, the contents of the
list items are largely technology companies so some inference can be
drawn about trust in technology companies versus the government.

The results of the list experiment are shown in
Table~\ref{tbl-listexp.summary.data}. For respondents shown the
sensitive list item, one can estimate that about 60\% of people selected
it (given that the baseline level is about 0.6 number of items selected
lower than compared to when respondents are shown the sensitive list
item). While it is hard to directly compare with the Likert-scaled
questions, note that the average on the Likert scale questions for
various trust measure of government use of data was about 4.5 out of 7,
or roughly the 65th percentile of the scale. For private corporations,
consider that 3 out of the 4 list items were private technology firms
and the other list item was their family. If one speculatively assumes
that most people will select the trust in their family list item, that
roughly indicates that respondents selected little over 1 out of 3 of
the private corporations as something that they trust. This roughly
accords with the average responses to trust in private corporations of
2.5 out of 7 on a Likert scale. These results are not meant to
definitively confirm that there are no issues of preference
falsification. That being said, the results do strongly parallel the
results of the regular survey questions, adding confidence to the
interpretation of the results of the previous tables.

\hypertarget{tbl-listexp.q.text}{}
\begin{table}
\caption{\label{tbl-listexp.q.text}List experiment questions }\tabularnewline

\centering
\begin{tabular}{l|>{\raggedright\arraybackslash}p{5in}}
\hline
Q1 & For the question below, please count how many of the entities listed below you would trust with your online personal information, such as details about your purchase history, your browsing habits, and your social media posts

• Alibaba

• Tencent

• Foreign internet companies (such as Microsoft)

• Your family

--> Sensitive item only shown to 50 per cent of respondents

• The central government\\
\hline
Q2 & For the question below, please count how many of the entities listed below you would trust with your online personal information, such as details about your purchase history, your browsing habits, and your social media posts

•   Alibaba

•   Tencent

•   Foreign internet companies (such as Microsoft)

•   Your family

--> Sensitive item only shown to 50 per cent of respondents>

•   The local government\\
\hline
\end{tabular}
\end{table}

\begin{table}

\caption{\label{tbl-listexp.summary.data}List experiment summary
data}\begin{minipage}[t]{\linewidth}
\subcaption{\label{tbl-listexp.summary.data-1}Central government list experiment}

{\centering 

\centering
\begin{tabular}[t]{lrrrr}
\toprule
\multicolumn{1}{c}{ } & \multicolumn{2}{c}{2021} & \multicolumn{2}{c}{2023} \\
\cmidrule(l{3pt}r{3pt}){2-3} \cmidrule(l{3pt}r{3pt}){4-5}
  & SI not shown & SI shown & SI not shown & SI shown\\
\midrule
\cellcolor{gray!6}{Number of items selected} & \cellcolor{gray!6}{\num{2.19}} & \cellcolor{gray!6}{\num{2.84}} & \cellcolor{gray!6}{\num{2.18}} & \cellcolor{gray!6}{\num{2.84}}\\
\bottomrule
\multicolumn{5}{l}{\rule{0pt}{1em}SI = sensitive item}\\
\end{tabular}

}

\end{minipage}%
\newline
\begin{minipage}[t]{\linewidth}
\subcaption{\label{tbl-listexp.summary.data-2}Local government list experiment}

{\centering 

\centering
\begin{tabular}[t]{lrrrr}
\toprule
\multicolumn{1}{c}{ } & \multicolumn{2}{c}{2021} & \multicolumn{2}{c}{2023} \\
\cmidrule(l{3pt}r{3pt}){2-3} \cmidrule(l{3pt}r{3pt}){4-5}
  & SI not shown & SI shown & SI not shown & SI shown\\
\midrule
\cellcolor{gray!6}{Number of items selected} & \cellcolor{gray!6}{\num{2.25}} & \cellcolor{gray!6}{\num{2.86}} & \cellcolor{gray!6}{\num{2.24}} & \cellcolor{gray!6}{\num{2.77}}\\
\bottomrule
\multicolumn{5}{l}{\rule{0pt}{1em}SI = sensitive item}\\
\end{tabular}

}

\end{minipage}%

\end{table}

\hypertarget{sec-analysis}{%
\section{Additional analysis}\label{sec-analysis}}

In order to help rule out the possibility that some of the differences
observed between 2021 and 2023 in the government trust and
comfortability of being monitored are driven simply by demographic
changes, Table~\ref{tbl-basicreg} reports on the results of a simple
regression framework. The response variables are: 1) trust in the
central government, 2) trust in the local government, 3) comfort with
central government monitoring, 4) comfort with local government
monitoring, and 5) comfort with private company monitoring. Included in
the regression are a standard suite of demographic variables and the
year variable.

The year variable magnitude matches almost exactly the simple difference
in means observed in Section~\ref{sec-datasummary}. In terms of
interesting coefficients, age is consistenly positively related to
acceptance of government monitoring and negatively related to private
corporation monitoring. This finding may partially explain events like
the White Paper Revolution, which was primarily a protest of the young.
However, the magnitude of coefficient is not very impactful - changing a
respondent's age from 20 to 70 changes the predicted response to the
acceptance of monitoring questions by 0.5 points - less than the size of
the year coefficient (0.6). Not being a party member also decreased
acceptance of monitoring by about 0.2 points. Income is sometimes a
relevant predictor though inconsistently so. Overall, the regressions
have a very low \(r^2\), indicating that most of the variation in
individual responses is due to factors outside of demographic variables.
So, while suggestive, the demographic variables have only limited
substantive relationship to variation in attitudes on these questions.

\hypertarget{tbl-basicreg}{}
\begin{table}
\caption{\label{tbl-basicreg}Regressions on individual question results }\tabularnewline

\centering
\begin{tabular}[t]{lccccc}
\toprule
  & CG Trust & LG Trust & CG Monitor & LG Monitor & PR Monitor\\
\midrule
(Intercept) & \num{5.813}*** & \num{5.364}*** & \num{4.256}*** & \num{4.257}*** & \num{3.640}***\\
 & (\num{0.174}) & (\num{0.186}) & (\num{0.194}) & (\num{0.195}) & (\num{0.236})\\
Age & \num{0.002} & \num{0.000} & \num{0.009}*** & \num{0.010}*** & \num{-0.007}**\\
 & (\num{0.002}) & (\num{0.002}) & (\num{0.002}) & (\num{0.002}) & (\num{0.003})\\
Middle school & \num{-0.132} & \num{-0.240}+ & \num{-0.042} & \num{-0.069} & \num{-0.121}\\
 & (\num{0.122}) & (\num{0.130}) & (\num{0.135}) & (\num{0.136}) & (\num{0.165})\\
High school & \num{-0.032} & \num{-0.215}+ & \num{0.101} & \num{0.074} & \num{-0.158}\\
 & (\num{0.114}) & (\num{0.121}) & (\num{0.127}) & (\num{0.127}) & \vphantom{1} (\num{0.154})\\
University & \num{-0.023} & \num{-0.189} & \num{0.192} & \num{0.202} & \num{-0.149}\\
 & (\num{0.114}) & (\num{0.121}) & (\num{0.127}) & (\num{0.127}) & (\num{0.154})\\
Grad school & \num{-0.130} & \num{-0.175} & \num{0.018} & \num{0.105} & \num{-0.166}\\
 & (\num{0.143}) & (\num{0.152}) & (\num{0.159}) & (\num{0.159}) & (\num{0.193})\\
Income 3000-5999 & \num{0.276}** & \num{0.272}** & \num{0.022} & \num{0.097} & \num{-0.084}\\
 & (\num{0.089}) & (\num{0.095}) & (\num{0.100}) & (\num{0.100}) & (\num{0.121})\\
Income 6000-9999 & \num{0.258}** & \num{0.242}* & \num{0.115} & \num{0.148} & \num{-0.160}\\
 & (\num{0.090}) & (\num{0.096}) & (\num{0.100}) & (\num{0.100}) & (\num{0.122})\\
Income 10000-19999 & \num{0.288}** & \num{0.316}** & \num{0.184}+ & \num{0.161} & \num{-0.437}***\\
 & (\num{0.091}) & (\num{0.097}) & (\num{0.102}) & (\num{0.102}) & (\num{0.124})\\
Income 20000-49999 & \num{0.285}** & \num{0.341}** & \num{0.188} & \num{0.171} & \num{-0.501}***\\
 & (\num{0.107}) & (\num{0.114}) & (\num{0.119}) & (\num{0.119}) & (\num{0.145})\\
Income 50000-99999 & \num{0.237} & \num{0.362}* & \num{0.253} & \num{0.008} & \num{-0.107}\\
 & (\num{0.156}) & (\num{0.166}) & (\num{0.174}) & (\num{0.174}) & (\num{0.211})\\
Income > 100000 & \num{0.189} & \num{0.166} & \num{-0.034} & \num{-0.051} & \num{-0.061}\\
 & (\num{0.173}) & (\num{0.184}) & (\num{0.193}) & (\num{0.193}) & (\num{0.234})\\
Male & \num{0.034} & \num{-0.003} & \num{-0.075} & \num{-0.092}+ & \num{-0.010}\\
 & (\num{0.043}) & (\num{0.046}) & (\num{0.048}) & (\num{0.048}) & (\num{0.058})\\
Not a party member & \num{-0.112}+ & \num{-0.038} & \num{-0.215}** & \num{-0.186}* & \num{0.020}\\
 & (\num{0.065}) & (\num{0.069}) & (\num{0.073}) & (\num{0.073}) & (\num{0.088})\\
Location: small city & \num{0.040} & \num{0.132}+ & \num{0.078} & \num{0.017} & \num{-0.129}\\
 & (\num{0.072}) & (\num{0.076}) & (\num{0.080}) & (\num{0.080}) & (\num{0.097})\\
Location: mid city & \num{-0.030} & \num{0.233}** & \num{0.073} & \num{0.019} & \num{-0.156}\\
 & (\num{0.076}) & (\num{0.081}) & (\num{0.085}) & (\num{0.085}) & (\num{0.103})\\
Location: big city & \num{-0.151}* & \num{0.176}* & \num{0.083} & \num{0.046} & \num{-0.079}\\
 & (\num{0.075}) & (\num{0.080}) & (\num{0.084}) & (\num{0.084}) & (\num{0.102})\\
Year 2023 & \num{-0.245}*** & \num{-0.187}*** & \num{0.612}*** & \num{0.402}*** & \num{0.150}*\\
 & (\num{0.044}) & (\num{0.047}) & (\num{0.049}) & (\num{0.049}) & (\num{0.059})\\
\midrule
Num.Obs. & \num{3507} & \num{3507} & \num{3507} & \num{3507} & \num{3507}\\
R2 & \num{0.018} & \num{0.015} & \num{0.055} & \num{0.030} & \num{0.017}\\
\bottomrule
\multicolumn{6}{l}{\rule{0pt}{1em}+ p $<$ 0.1, * p $<$ 0.05, ** p $<$ 0.01, *** p $<$ 0.001}\\
\multicolumn{6}{l}{\rule{0pt}{1em}Reference categories are Less than middle, Income 0-2999, Female, Party member, Countryside}\\
\end{tabular}
\end{table}

Turning toward differences between government and private protection of
privacy, Table~\ref{tbl-trackingreg} presents regression results for
tracking questions on the same suite of demographic controls. In
particular, these regressions consider the question of ``how closely do
you think \emph{x} tracks your online activity?'' (regressions 1-3) and
``How comfortable are you with \emph{x} knowing personal details about
your activity online?'' (regressions 4-6). Again, it should be repeated
that the \(r^2\) for all regressions are also very low - most variation
in question response is not accounted for in the model. However, there
are a few interesting observations to make. First is that those in
larger cities are more likely to notice monitoring as compared to those
living small cities, even controlling for income and education. The
magnitude of the coefficient is not large (0.3 for those living in big
cities) but does suggest that urban surveillance is qualitatively of a
different type or scale. While the year coefficient is significant in
some instances, the magnitude is very small (0.1) indicating a
substantively insignificant effect. Interestingly, not being a party
member is associated both with less noticing of motioning and less
comfort with organizations knowing their private details. Again, the
coefficients are relatively small but does suggest that having some
level of knowledge of what kind of surveillance is associated with more
positive views of its use. Some of the other demographic variables are
occasionally significant but not in a way that indicates a consistent
and meaningful relationship with the response variables.

\hypertarget{tbl-trackingreg}{}
\begin{table}
\caption{\label{tbl-trackingreg}Regressions on government vs.~private tracking }\tabularnewline

\centering
\begin{tabular}[t]{lcccccc}
\toprule
  & CG Track & LG Track & PR Track & CG PD & LG PD & PR PD\\
\midrule
(Intercept) & \num{4.121}*** & \num{4.009}*** & \num{3.913}*** & \num{4.247}*** & \num{4.209}*** & \num{3.481}***\\
 & (\num{0.188}) & (\num{0.188}) & (\num{0.224}) & (\num{0.214}) & (\num{0.216}) & (\num{0.238})\\
Age & \num{0.000} & \num{0.002} & \num{0.002} & \num{0.004}+ & \num{0.002} & \num{-0.007}*\\
 & (\num{0.002}) & (\num{0.002}) & (\num{0.003}) & (\num{0.003}) & (\num{0.003}) & (\num{0.003})\\
Middle school & \num{-0.066} & \num{-0.085} & \num{0.146} & \num{-0.143} & \num{-0.013} & \num{-0.033}\\
 & (\num{0.131}) & (\num{0.131}) & (\num{0.156}) & (\num{0.149}) & (\num{0.151}) & (\num{0.166})\\
High school & \num{0.021} & \num{0.010} & \num{0.037} & \num{-0.062} & \num{0.046} & \num{-0.059}\\
 & (\num{0.123}) & (\num{0.123}) & (\num{0.147}) & (\num{0.140}) & (\num{0.141}) & (\num{0.155})\\
University & \num{0.006} & \num{0.019} & \num{0.129} & \num{-0.049} & \num{0.046} & \num{-0.169}\\
 & (\num{0.123}) & (\num{0.123}) & (\num{0.146}) & (\num{0.140}) & (\num{0.141}) & (\num{0.155})\\
Grad school & \num{0.218} & \num{0.259}+ & \num{0.244} & \num{-0.081} & \num{-0.082} & \num{0.088}\\
 & (\num{0.154}) & (\num{0.154}) & (\num{0.184}) & (\num{0.176}) & (\num{0.177}) & (\num{0.195})\\
Income 3000-5999 & \num{-0.013} & \num{0.015} & \num{-0.017} & \num{0.026} & \num{-0.052} & \num{-0.117}\\
 & (\num{0.096}) & (\num{0.097}) & (\num{0.115}) & (\num{0.110}) & (\num{0.111}) & (\num{0.122})\\
Income 6000-9999 & \num{0.073} & \num{0.035} & \num{0.186} & \num{0.081} & \num{-0.046} & \num{-0.045}\\
 & (\num{0.097}) & (\num{0.097}) & (\num{0.116}) & (\num{0.111}) & (\num{0.112}) & (\num{0.123})\\
Income 10000-19999 & \num{0.069} & \num{0.064} & \num{0.244}* & \num{-0.006} & \num{-0.117} & \num{-0.288}*\\
 & (\num{0.099}) & (\num{0.099}) & (\num{0.118}) & (\num{0.112}) & (\num{0.113}) & (\num{0.125})\\
Income 20000-49999 & \num{0.129} & \num{0.136} & \num{0.384}** & \num{0.082} & \num{-0.177} & \num{-0.344}*\\
 & (\num{0.115}) & (\num{0.115}) & (\num{0.137}) & (\num{0.131}) & (\num{0.132}) & (\num{0.146})\\
Income 50000-99999 & \num{0.290}+ & \num{0.134} & \num{0.554}** & \num{-0.016} & \num{-0.067} & \num{-0.085}\\
 & (\num{0.168}) & (\num{0.168}) & (\num{0.201}) & (\num{0.192}) & (\num{0.193}) & (\num{0.213})\\
Income > 100000 & \num{0.146} & \num{0.038} & \num{0.320} & \num{-0.141} & \num{-0.198} & \num{-0.196}\\
 & (\num{0.187}) & (\num{0.187}) & (\num{0.222}) & (\num{0.212}) & (\num{0.214}) & (\num{0.236})\\
Male & \num{0.012} & \num{-0.092}* & \num{0.184}*** & \num{0.010} & \num{0.057} & \num{0.103}+\\
 & (\num{0.046}) & (\num{0.047}) & (\num{0.055}) & (\num{0.053}) & (\num{0.053}) & (\num{0.059})\\
Not a party member & \num{-0.125}+ & \num{-0.090} & \num{-0.178}* & \num{-0.203}* & \num{-0.227}** & \num{-0.220}*\\
 & (\num{0.070}) & (\num{0.070}) & (\num{0.084}) & (\num{0.080}) & (\num{0.081}) & (\num{0.089})\\
Location: small city & \num{0.089} & \num{0.096} & \num{0.070} & \num{-0.022} & \num{-0.058} & \num{-0.222}*\\
 & (\num{0.077}) & (\num{0.077}) & (\num{0.092}) & (\num{0.088}) & (\num{0.089}) & (\num{0.098})\\
Location: mid city & \num{0.187}* & \num{0.218}** & \num{0.145} & \num{0.065} & \num{0.012} & \num{-0.253}*\\
 & (\num{0.082}) & (\num{0.082}) & (\num{0.098}) & (\num{0.093}) & (\num{0.094}) & (\num{0.104})\\
Location: big city & \num{0.356}*** & \num{0.335}*** & \num{0.270}** & \num{0.072} & \num{0.086} & \num{-0.170}+\\
 & (\num{0.081}) & (\num{0.081}) & (\num{0.097}) & (\num{0.093}) & (\num{0.093}) & (\num{0.103})\\
Year 2023 & \num{0.112}* & \num{0.148}** & \num{-0.014} & \num{-0.083} & \num{-0.041} & \num{-0.131}*\\
 & (\num{0.047}) & (\num{0.047}) & (\num{0.056}) & (\num{0.054}) & (\num{0.054}) & (\num{0.060})\\
\midrule
Num.Obs. & \num{3507} & \num{3507} & \num{3507} & \num{3507} & \num{3507} & \num{3507}\\
R2 & \num{0.020} & \num{0.020} & \num{0.026} & \num{0.006} & \num{0.005} & \num{0.014}\\
\bottomrule
\multicolumn{7}{l}{\rule{0pt}{1em}+ p $<$ 0.1, * p $<$ 0.05, ** p $<$ 0.01, *** p $<$ 0.001}\\
\multicolumn{7}{l}{\rule{0pt}{1em}Reference categories are Less than middle, Income 0-2999, Female, Party member, Countryside}\\
\end{tabular}
\end{table}

Finally, a look at the subset of the data for those living in Xi'an,
Shanghai, and Wuhan. Both Shanghai and Xi'an suffered painful lockdowns
in December of 2021 and April 2022, respectively. Wuhan was the original
source of the Covid outbreak and underwent a many months long set of
lockdwns and restrictions in 2021. The results for key questions broken
down by city can be found in Table~\ref{tbl-gp.detailed.data}. The set
of questions relating to overall trust and whether respondents see good
reasons to allow each entity to collect their data (part (a)) suggests
that the only city that is significantly different from the overall
pattern of cities is Shanghai, but in ways that one might expect.
Shanghainese trust their local government more than cities overall (the
Shanghai coefficient on LG Trust). This is not a surprising result given
that Shanghai is one of the most developed cities in China. However, the
results also seem to indicate that Shanghai residents saw a very
substantively large decrease in trust in their local governments (the
Shanghai x 2023 coefficient on CG Trust and LG Trust). Furthermore
Shanghai residents also had a major decrease in agreement that the
government has a good reason to monitor them in 2023 (the Shanghai x
2023 coefficient on CG Monitor and LG monitor). The size of the
coefficient on all of these is also substantively very large. None of
the other cities seem significantly different from the average of other
cities or locations.

These results make sense given the severity of the Shanghai lockdown and
the post-lockdown strict controls. For many residents, the lockdown was
highly traumatic but without a sense of shared unity or purpose, as was
the case in the 2020 Wuhan lockdown. Furthermore, the Shaghainese local
government seriously mishandled the logistics of the lockdown, leaving
many people scrambling for food and medicine. However, the results
presented in part (b) complicate this story somewhat. Shanghai does not
appear to be any different than other cities while respondents from
Wuhan consistently are more likely to notice tracking efforts. I have no
stong hypothesis as to the Wuhan results but the Shanghai findings may
be suggestive that respondents are reading this set of questions as
referring to the present rather than including the pandemic time period.

\begin{table}

\caption{\label{tbl-gp.detailed.data}Key questions by
city}\begin{minipage}[t]{\linewidth}
\subcaption{\label{tbl-gp.detailed.data-1}Trust questions }

{\centering 

\tabularnewline

\centering
\begin{tabular}[t]{lccccc}
\toprule
  & CG Trust & LG Trust & CG Monitor & LG Monitor & PR Monitor\\
\midrule
(Intercept) & \num{5.994}*** & \num{5.541}*** & \num{4.673}*** & \num{4.663}*** & \num{2.919}***\\
 & (\num{0.036}) & (\num{0.038}) & (\num{0.040}) & (\num{0.040}) & (\num{0.049})\\
Shanghai & \num{0.163} & \num{0.371}* & \num{0.292} & \num{0.214} & \num{-0.235}\\
 & (\num{0.172}) & (\num{0.183}) & (\num{0.192}) & (\num{0.192}) & (\num{0.234})\\
Xi'an & \num{0.052} & \num{0.063} & \num{0.025} & \num{-0.082} & \num{0.151}\\
 & (\num{0.197}) & (\num{0.210}) & (\num{0.220}) & (\num{0.220}) & (\num{0.268})\\
Wuhan & \num{-0.173} & \num{-0.071} & \num{-0.169} & \num{-0.133} & \num{-0.078}\\
 & (\num{0.109}) & (\num{0.117}) & (\num{0.122}) & (\num{0.122}) & (\num{0.149})\\
2023 & \num{-0.227}*** & \num{-0.190}*** & \num{0.608}*** & \num{0.405}*** & \num{0.194}**\\
 & (\num{0.047}) & (\num{0.051}) & (\num{0.053}) & (\num{0.053}) & (\num{0.065})\\
Shanghai x 2023 & \num{-0.634}** & \num{-0.613}** & \num{-0.633}** & \num{-0.559}* & \num{-0.007}\\
 & (\num{0.215}) & (\num{0.230}) & (\num{0.241}) & (\num{0.241}) & (\num{0.293})\\
Xi'an x 2023 & \num{-0.045} & \num{-0.092} & \num{0.081} & \num{0.079} & \num{-0.264}\\
 & (\num{0.303}) & (\num{0.323}) & (\num{0.339}) & (\num{0.339}) & (\num{0.412})\\
Wuhan x 2023 & \num{0.064} & \num{0.161} & \num{0.047} & \num{0.006} & \num{-0.085}\\
 & (\num{0.145}) & (\num{0.154}) & (\num{0.162}) & (\num{0.162}) & (\num{0.197})\\
\bottomrule
\multicolumn{6}{l}{\rule{0pt}{1em}+ p $<$ 0.1, * p $<$ 0.05, ** p $<$ 0.01, *** p $<$ 0.001}\\
\end{tabular}

}

\end{minipage}%
\newline
\begin{minipage}[t]{\linewidth}
\subcaption{\label{tbl-gp.detailed.data-2}Data use questions }

{\centering 

\tabularnewline

\centering
\begin{tabular}[t]{lcccccc}
\toprule
  & CG Track & LG Track & PR Track & CG PD & LG PD & PR PD\\
\midrule
(Intercept) & \num{4.264}*** & \num{4.192}*** & \num{4.360}*** & \num{4.187}*** & \num{4.054}*** & \num{2.649}***\\
 & (\num{0.039}) & (\num{0.039}) & (\num{0.047}) & (\num{0.044}) & (\num{0.044}) & (\num{0.049})\\
Shanghai & \num{0.490}** & \num{0.439}* & \num{0.166} & \num{0.146} & \num{0.209} & \num{-0.088}\\
 & (\num{0.186}) & (\num{0.186}) & (\num{0.223}) & (\num{0.211}) & (\num{0.213}) & (\num{0.235})\\
Xi'an & \num{0.340} & \num{0.343} & \num{0.314} & \num{0.022} & \num{0.039} & \num{0.537}*\\
 & (\num{0.213}) & (\num{0.213}) & (\num{0.255}) & (\num{0.242}) & (\num{0.244}) & (\num{0.269})\\
Wuhan & \num{0.146} & \num{0.046} & \num{-0.009} & \num{0.243}+ & \num{0.237}+ & \num{-0.179}\\
 & (\num{0.118}) & (\num{0.119}) & (\num{0.142}) & (\num{0.134}) & (\num{0.135}) & (\num{0.150})\\
2023 & \num{0.079} & \num{0.112}* & \num{-0.102}+ & \num{-0.064} & \num{-0.009} & \num{-0.125}+\\
 & (\num{0.051}) & (\num{0.051}) & (\num{0.062}) & (\num{0.058}) & (\num{0.059}) & (\num{0.065})\\
Shanghai x 2023 & \num{-0.289} & \num{-0.259} & \num{0.180} & \num{-0.279} & \num{-0.254} & \num{0.039}\\
 & (\num{0.233}) & (\num{0.234}) & (\num{0.280}) & (\num{0.265}) & (\num{0.267}) & (\num{0.295})\\
Xi'an x 2023 & \num{-0.232} & \num{-0.453} & \num{-0.153} & \num{0.371} & \num{0.077} & \num{-0.609}\\
 & (\num{0.328}) & (\num{0.328}) & (\num{0.393}) & (\num{0.372}) & (\num{0.375}) & (\num{0.414})\\
Wuhan x 2023 & \num{0.273}+ & \num{0.348}* & \num{0.444}* & \num{-0.153} & \num{-0.223} & \num{0.328}+\\
 & (\num{0.157}) & (\num{0.157}) & (\num{0.188}) & (\num{0.178}) & (\num{0.179}) & (\num{0.198})\\
\bottomrule
\multicolumn{7}{l}{\rule{0pt}{1em}+ p $<$ 0.1, * p $<$ 0.05, ** p $<$ 0.01, *** p $<$ 0.001}\\
\end{tabular}

}

\end{minipage}%

\end{table}

\hypertarget{discussion-and-conclusion}{%
\section{Discussion and conclusion}\label{discussion-and-conclusion}}

With respect to the hypotheses posed in Section~\ref{sec-litreview}, the
only one that found unambiguous support was \(H_2\), that Chinese
respondents trust private corporations with their digital data at a
significantly lower level than the government. This survey did find that
overall trust in government decreased somewhat between 2021 and 2023 but
there was no corresponding decrease in acceptance of government
surveillance (as would be expected from \(H_1\), \(H_3\), and \(H_4\)).
In fact, if anything, acceptance of the principle of government
monitoring increased. The only exception to this pattern is the
respondents located in Shanghai, which did see a significant decrease in
government trust and acceptance of surveillance (as predicted by the
hypotheses). A plausible interpretation of these results is that the
average Covid-19 control experience for most Chinese citizens was
relatively mild and did not significantly change their views of the
state or of digital surveillance. Only in a place where the surveillance
and controls were especially severe, like Shanghai, do we notice a
significant shift in attitudes. It may be that if the relatively more
mild experience were extended for a longer period of time elsewhere in
China, survey responses would show a more significant attitudinal
change. Overall, it suggests that acceptance of government monitoring
more or less follows public trust in the government but is also impacted
by the reason and need for the surveillance, a factor not given
sustained attention in the literature. Contra stories of a new
surveillance dystopia in China, this research suggests that even
relatively invasive monitoring will not lead to significant changes in
attitudes in China.

\newpage{}

\hypertarget{references}{%
\section{References}\label{references}}

\hypertarget{refs}{}
\begin{CSLReferences}{1}{0}
\leavevmode\vadjust pre{\hypertarget{ref-ang2022}{}}%
Ang YY (2022)
\href{https://www.foreignaffairs.com/china/problem-zero-xi-pandemic-policy-crisis}{The
problem with zero}. \emph{Foreign Affairs}. Epub ahead of print 2
December 2022.

\leavevmode\vadjust pre{\hypertarget{ref-wuhanlo2021}{}}%
\emph{BBC News} (2021)
\href{https://www.bbc.com/news/world-asia-china-55628488}{Wuhan
lockdown: A year of china's fight against the covid pandemic}. Epub
ahead of print 22 January 2021.

\leavevmode\vadjust pre{\hypertarget{ref-blair2012}{}}%
Blair G and Imai K (2012)
\href{https://doi.org/10.1093/pan/mpr048}{Statistical Analysis of List
Experiments}. \emph{Political Analysis} 20(1): 47--77.

\leavevmode\vadjust pre{\hypertarget{ref-chen2017}{}}%
Chen D (2017) \href{https://doi.org/10.1177/1065912917691360}{Local
Distrust and Regime Support: Sources and Effects of Political Trust in
China}. \emph{Political Research Quarterly} 70(2): 314--326.

\leavevmode\vadjust pre{\hypertarget{ref-dou2022}{}}%
Dou E (2022)
\href{https://www.washingtonpost.com/world/2022/05/24/china-covid-lockdown-testing-green/}{It{'}s
not easy staying green: Keeping out of china{'}s covid lockdown}.
\emph{Washington Post}. Epub ahead of print 24 May 2022.

\leavevmode\vadjust pre{\hypertarget{ref-dou2022a}{}}%
Dou E, Qiang V and Li L (2022)
\href{https://www.washingtonpost.com/world/2022/04/01/china-shanghai-medical-emergencies-coronavirus-lockdown/}{Medical
emergencies mount as shanghai{'}s lockdown tightens}. \emph{Washington
Post}. Epub ahead of print 1 April 2022.

\leavevmode\vadjust pre{\hypertarget{ref-gainous2023}{}}%
Gainous J, Han R, MacDonald AW, et al. (2023) \emph{Directed Digital
Dissidence in Autocracies: How China Wins Online}. Oxford University
Press. Available at: \url{https://academic.oup.com/book/55291}.

\leavevmode\vadjust pre{\hypertarget{ref-huang2022}{}}%
Huang K and Han M (2022) Did China{'}s Street Protests End Harsh COVID
Policies? Available at:
\url{https://www.cfr.org/blog/did-chinas-street-protests-end-harsh-covid-policies}.

\leavevmode\vadjust pre{\hypertarget{ref-intercepted2022}{}}%
Intercepted (2022) Intercepted: Inside china{'}s growing surveillance
state. Available at:
\url{https://theintercept.com/2022/10/05/intercepted-china-surveillance/}.

\leavevmode\vadjust pre{\hypertarget{ref-ioannou2021}{}}%
Ioannou A and Tussyadiah I (2021)
\href{https://doi.org/10.1016/j.techsoc.2021.101774}{Privacy and
surveillance attitudes during health crises: Acceptance of surveillance
and privacy protection behaviours}. \emph{Technology in Society} 67:
101774.

\leavevmode\vadjust pre{\hypertarget{ref-kennedy2009}{}}%
Kennedy JJ (2009)
\href{https://doi.org/10.1111/j.1467-9248.2008.00740.x}{Maintaining
Popular Support for the Chinese Communist Party: The Influence of
Education and the State-Controlled Media}. \emph{Political Studies}
57(3): 517--536.

\leavevmode\vadjust pre{\hypertarget{ref-king2013}{}}%
King G, Pan J and Roberts ME (2013) How censorship in china allows
government criticism but silences collective expression. \emph{American
Political Science Review} 107: 1--18.

\leavevmode\vadjust pre{\hypertarget{ref-kostka2019}{}}%
Kostka G (2019)
\href{https://doi.org/10.1177/1461444819826402}{China{'}s social credit
systems and public opinion: Explaining high levels of approval}.
\emph{New Media \& Society} 21(7): 1565--1593.

\leavevmode\vadjust pre{\hypertarget{ref-lin2022}{}}%
Lin L and Jie Y (2022)
\href{https://www.wsj.com/articles/shanghai-lockdown-leads-to-logistics-disarray-with-quarantined-truckers-piled-up-containers-11650537303}{Shanghai{'}s
covid lockdown leads to logistics disarray, with quarantined truckers,
piled-up containers}. \emph{Wall Street Journal}. Epub ahead of print 21
April 2022.

\leavevmode\vadjust pre{\hypertarget{ref-mao2022}{}}%
Mao (2022)
\href{https://www.bbc.com/news/world-asia-china-63855508}{China abandons
key parts of zero-covid strategy after protests}. \emph{BBC News}. Epub
ahead of print 7 December 2022.

\leavevmode\vadjust pre{\hypertarget{ref-mcmorrow2022}{}}%
McMorrow R and Leng C (2022)
\href{https://www.ft.com/content/dee6bcc6-3fc5-4edc-814d-46dc73e67c7e}{{`}Digital
handcuffs{'}: China{'}s covid health apps govern life but are ripe for
abuse}. \emph{Financial Times}. Epub ahead of print 28 June 2022.

\leavevmode\vadjust pre{\hypertarget{ref-moseson2017}{}}%
Moseson H, Treleaven E, Gerdts C, et al. (2017)
\href{https://doi.org/10.1111/sifp.12042}{The List Experiment for
Measuring Abortion: What We Know and What We Need}. \emph{Studies in
Family Planning} 48(4): 397--405.

\leavevmode\vadjust pre{\hypertarget{ref-baiduch2018}{}}%
\emph{People's Daily Online} (2018)
\href{http://en.people.cn/n3/2018/0328/c90000-9442509.html}{Baidu chief
under fire for privacy comments}. Epub ahead of print 28 March 2018.

\leavevmode\vadjust pre{\hypertarget{ref-pfaff2001}{}}%
PFAFF S (2001) \href{https://doi.org/10.1177/1462474501003003003}{The
Limits of Coercive Surveillance: Social and Penal Control in the German
Democratic Republic}. \emph{Punishment \& Society} 3(3): 381--407.

\leavevmode\vadjust pre{\hypertarget{ref-reddick2015}{}}%
Reddick CG, Chatfield AT and Jaramillo PA (2015)
\href{https://doi.org/10.1016/j.giq.2015.01.003}{Public opinion on
national security agency surveillance programs: A multi-method
approach}. \emph{Government Information Quarterly} 32(2): 129--141.

\leavevmode\vadjust pre{\hypertarget{ref-redlawsk2010}{}}%
Redlawsk DP, Tolbert CJ and Franko W (2010)
\href{https://doi.org/10.1177/1065912910373554}{Voters, Emotions, and
Race in 2008: Obama as the First Black President}. \emph{Political
Research Quarterly} 63(4): 875--889.

\leavevmode\vadjust pre{\hypertarget{ref-shen2018}{}}%
Shen X (2018)
\href{https://www.scmp.com/abacus/tech/article/3028402/chinese-internet-users-criticize-baidu-ceo-saying-people-china-are}{Chinese
internet users criticize Baidu CEO for saying people in China are
willing to give up data privacy for convenience}. \emph{South China
Morning Post}. Epub ahead of print 28 March 2018.

\leavevmode\vadjust pre{\hypertarget{ref-steinhardt2022}{}}%
Steinhardt HC, Holzschuh L and MacDonald AW (2022) Dreading big brother
or dreading big profit? Privacy concerns toward the state and companies
in China. \emph{First Monday}. Epub ahead of print 13 December 2022.
DOI:
\href{https://doi.org/10.5210/fm.v27i12.12679}{10.5210/fm.v27i12.12679}.

\leavevmode\vadjust pre{\hypertarget{ref-traunmuxfcller2019}{}}%
Traunmüller R, Kijewski S and Freitag M (2019)
\href{https://doi.org/10.1177/0022002719828053}{The Silent Victims of
Sexual Violence during War: Evidence from a List Experiment in Sri
Lanka}. \emph{Journal of Conflict Resolution} 63(9): 2015--2042.

\leavevmode\vadjust pre{\hypertarget{ref-truxfcdinger2017}{}}%
Trüdinger E-M and Steckermeier LC (2017)
\href{https://doi.org/10.1016/j.giq.2017.07.003}{Trusting and
controlling? Political trust, information and acceptance of surveillance
policies: The case of germany}. \emph{Government Information Quarterly}
34(3): 421--433.

\leavevmode\vadjust pre{\hypertarget{ref-vlahos2019}{}}%
Vlahos KB (2019)
\href{https://www.theamericanconservative.com/george-orwells-dystopian-nightmare-in-china-1984/}{George
orwell{'}s dystopian nightmare in china}. \emph{The American
Conservative}. Epub ahead of print 24 June 2019.

\leavevmode\vadjust pre{\hypertarget{ref-wang2015}{}}%
Wang Z and Yu Q (2015)
\href{https://doi.org/10.1016/j.clsr.2015.08.006}{Privacy trust crisis
of personal data in china in the era of big data: The survey and
countermeasures}. \emph{Computer Law \& Security Review} 31(6):
782--792.

\leavevmode\vadjust pre{\hypertarget{ref-xu2021}{}}%
Xu X (2021) \href{https://doi.org/10.1111/ajps.12514}{To Repress or to
Co-opt? Authoritarian Control in the Age of Digital Surveillance}.
\emph{American Journal of Political Science} 65(2): 309--325.

\leavevmode\vadjust pre{\hypertarget{ref-zhang2002}{}}%
Zhang Y``Jeff'', Chen JQ and Wen K-W (2002)
\href{https://doi.org/10.1300/J179v01n02_01}{Characteristics of internet
users and their privacy concerns}. \emph{Journal of Internet Commerce}
1(2): 1--16.

\leavevmode\vadjust pre{\hypertarget{ref-zhong2014}{}}%
Zhong Y (2014) Do Chinese People Trust Their Local Government, and Why?
\emph{Problems of Post-Communism}. Epub ahead of print 1 May 2014. DOI:
\href{https://doi.org/10.2753/PPC1075-8216610303}{10.2753/PPC1075-8216610303}.

\end{CSLReferences}



\end{document}
