% Options for packages loaded elsewhere
\PassOptionsToPackage{unicode}{hyperref}
\PassOptionsToPackage{hyphens}{url}
\PassOptionsToPackage{dvipsnames,svgnames,x11names}{xcolor}
%
\documentclass[
  letterpaper,
  DIV=11,
  numbers=noendperiod]{scrartcl}

\usepackage{amsmath,amssymb}
\usepackage{iftex}
\ifPDFTeX
  \usepackage[T1]{fontenc}
  \usepackage[utf8]{inputenc}
  \usepackage{textcomp} % provide euro and other symbols
\else % if luatex or xetex
  \usepackage{unicode-math}
  \defaultfontfeatures{Scale=MatchLowercase}
  \defaultfontfeatures[\rmfamily]{Ligatures=TeX,Scale=1}
\fi
\usepackage{lmodern}
\ifPDFTeX\else  
    % xetex/luatex font selection
\fi
% Use upquote if available, for straight quotes in verbatim environments
\IfFileExists{upquote.sty}{\usepackage{upquote}}{}
\IfFileExists{microtype.sty}{% use microtype if available
  \usepackage[]{microtype}
  \UseMicrotypeSet[protrusion]{basicmath} % disable protrusion for tt fonts
}{}
\makeatletter
\@ifundefined{KOMAClassName}{% if non-KOMA class
  \IfFileExists{parskip.sty}{%
    \usepackage{parskip}
  }{% else
    \setlength{\parindent}{0pt}
    \setlength{\parskip}{6pt plus 2pt minus 1pt}}
}{% if KOMA class
  \KOMAoptions{parskip=half}}
\makeatother
\usepackage{xcolor}
\setlength{\emergencystretch}{3em} % prevent overfull lines
\setcounter{secnumdepth}{-\maxdimen} % remove section numbering
% Make \paragraph and \subparagraph free-standing
\ifx\paragraph\undefined\else
  \let\oldparagraph\paragraph
  \renewcommand{\paragraph}[1]{\oldparagraph{#1}\mbox{}}
\fi
\ifx\subparagraph\undefined\else
  \let\oldsubparagraph\subparagraph
  \renewcommand{\subparagraph}[1]{\oldsubparagraph{#1}\mbox{}}
\fi


\providecommand{\tightlist}{%
  \setlength{\itemsep}{0pt}\setlength{\parskip}{0pt}}\usepackage{longtable,booktabs,array}
\usepackage{calc} % for calculating minipage widths
% Correct order of tables after \paragraph or \subparagraph
\usepackage{etoolbox}
\makeatletter
\patchcmd\longtable{\par}{\if@noskipsec\mbox{}\fi\par}{}{}
\makeatother
% Allow footnotes in longtable head/foot
\IfFileExists{footnotehyper.sty}{\usepackage{footnotehyper}}{\usepackage{footnote}}
\makesavenoteenv{longtable}
\usepackage{graphicx}
\makeatletter
\def\maxwidth{\ifdim\Gin@nat@width>\linewidth\linewidth\else\Gin@nat@width\fi}
\def\maxheight{\ifdim\Gin@nat@height>\textheight\textheight\else\Gin@nat@height\fi}
\makeatother
% Scale images if necessary, so that they will not overflow the page
% margins by default, and it is still possible to overwrite the defaults
% using explicit options in \includegraphics[width, height, ...]{}
\setkeys{Gin}{width=\maxwidth,height=\maxheight,keepaspectratio}
% Set default figure placement to htbp
\makeatletter
\def\fps@figure{htbp}
\makeatother
\newlength{\cslhangindent}
\setlength{\cslhangindent}{1.5em}
\newlength{\csllabelwidth}
\setlength{\csllabelwidth}{3em}
\newlength{\cslentryspacingunit} % times entry-spacing
\setlength{\cslentryspacingunit}{\parskip}
\newenvironment{CSLReferences}[2] % #1 hanging-ident, #2 entry spacing
 {% don't indent paragraphs
  \setlength{\parindent}{0pt}
  % turn on hanging indent if param 1 is 1
  \ifodd #1
  \let\oldpar\par
  \def\par{\hangindent=\cslhangindent\oldpar}
  \fi
  % set entry spacing
  \setlength{\parskip}{#2\cslentryspacingunit}
 }%
 {}
\usepackage{calc}
\newcommand{\CSLBlock}[1]{#1\hfill\break}
\newcommand{\CSLLeftMargin}[1]{\parbox[t]{\csllabelwidth}{#1}}
\newcommand{\CSLRightInline}[1]{\parbox[t]{\linewidth - \csllabelwidth}{#1}\break}
\newcommand{\CSLIndent}[1]{\hspace{\cslhangindent}#1}

\usepackage{booktabs}
\usepackage{longtable}
\usepackage{array}
\usepackage{multirow}
\usepackage{wrapfig}
\usepackage{float}
\usepackage{colortbl}
\usepackage{pdflscape}
\usepackage{tabu}
\usepackage{threeparttable}
\usepackage{threeparttablex}
\usepackage[normalem]{ulem}
\usepackage{makecell}
\usepackage{xcolor}
\usepackage{siunitx}

  \newcolumntype{d}{S[
    input-open-uncertainty=,
    input-close-uncertainty=,
    parse-numbers = false,
    table-align-text-pre=false,
    table-align-text-post=false
  ]}
  
\KOMAoption{captions}{tableheading}
\makeatletter
\makeatother
\makeatletter
\makeatother
\makeatletter
\@ifpackageloaded{caption}{}{\usepackage{caption}}
\AtBeginDocument{%
\ifdefined\contentsname
  \renewcommand*\contentsname{Table of contents}
\else
  \newcommand\contentsname{Table of contents}
\fi
\ifdefined\listfigurename
  \renewcommand*\listfigurename{List of Figures}
\else
  \newcommand\listfigurename{List of Figures}
\fi
\ifdefined\listtablename
  \renewcommand*\listtablename{List of Tables}
\else
  \newcommand\listtablename{List of Tables}
\fi
\ifdefined\figurename
  \renewcommand*\figurename{Figure}
\else
  \newcommand\figurename{Figure}
\fi
\ifdefined\tablename
  \renewcommand*\tablename{Table}
\else
  \newcommand\tablename{Table}
\fi
}
\@ifpackageloaded{float}{}{\usepackage{float}}
\floatstyle{ruled}
\@ifundefined{c@chapter}{\newfloat{codelisting}{h}{lop}}{\newfloat{codelisting}{h}{lop}[chapter]}
\floatname{codelisting}{Listing}
\newcommand*\listoflistings{\listof{codelisting}{List of Listings}}
\makeatother
\makeatletter
\@ifpackageloaded{caption}{}{\usepackage{caption}}
\@ifpackageloaded{subcaption}{}{\usepackage{subcaption}}
\makeatother
\makeatletter
\@ifpackageloaded{tcolorbox}{}{\usepackage[skins,breakable]{tcolorbox}}
\makeatother
\makeatletter
\@ifundefined{shadecolor}{\definecolor{shadecolor}{rgb}{.97, .97, .97}}
\makeatother
\makeatletter
\makeatother
\makeatletter
\makeatother
\ifLuaTeX
  \usepackage{selnolig}  % disable illegal ligatures
\fi
\IfFileExists{bookmark.sty}{\usepackage{bookmark}}{\usepackage{hyperref}}
\IfFileExists{xurl.sty}{\usepackage{xurl}}{} % add URL line breaks if available
\urlstyle{same} % disable monospaced font for URLs
\hypersetup{
  pdftitle={Learning to love big brother: Chinese attitudes toward online privacy after the pandemic},
  pdfauthor={Andrew MacDonald},
  pdfkeywords={Digital Privacy, Covid-19, Authoriatrian
Regimes, Government Trust, China},
  colorlinks=true,
  linkcolor={blue},
  filecolor={Maroon},
  citecolor={Blue},
  urlcolor={Blue},
  pdfcreator={LaTeX via pandoc}}

\title{Learning to love big brother: Chinese attitudes toward online
privacy after the pandemic}
\author{Andrew MacDonald}
\date{2023-10-08}

\begin{document}
\maketitle
\begin{abstract}
The results of the survey strongly suggest that, despite all of the
invasive government monitoring used to attempt to control the spread of
Covid, respondents were generally supportive of government invasions of
their privacy. These findings cast doubt on the long-term impact of the
White Paper movement protests and connect to a larger body of literature
on why surveys of Chinese citizens indicate high levels of trust in
their government.
\end{abstract}
\ifdefined\Shaded\renewenvironment{Shaded}{\begin{tcolorbox}[sharp corners, borderline west={3pt}{0pt}{shadecolor}, breakable, interior hidden, boxrule=0pt, frame hidden, enhanced]}{\end{tcolorbox}}\fi

\hypertarget{introduction}{%
\subsection{Introduction}\label{introduction}}

\hypertarget{literature-review}{%
\subsection{Literature review}\label{literature-review}}

\hypertarget{data-and-summary-statistics}{%
\subsection{Data and summary
statistics}\label{data-and-summary-statistics}}

The data for this project was collected via a commercial survey firm in
two waves, February of 2021 and March of 2023. In both the first and
second waves, Wuhan was oversampled, with residents of the city set to
be 10\% of respondents. The 2021 survey had an n=1500 and the second had
an n=2000. Questions on the two surveys were identical other than a
minor change to a question that referenced a specific date. The timing
of the two surveys came at two very different points in time of China's
Covid-19 experience. The first survey was conducted approximately seven
months after the last round of restrictions were lifted on the city of
Wuhan. China, at the time, was essentially closed to foreign travel but
otherwise had little in the way of day to day public health
restrictions. Nationwide, daily Covid cases hovered around the single
digits (\emph{BBC News}, 2021). China was at a very different point in
its journey in March of 2023. The year of 2022 saw widespread, intrusive
digital monitoring introduced. Many major cities, such as Shanghai,
Xi'an, and Shenzhen, underwent long and painful city-wide lockdown
procedures. At the end of 2023, under the weight of a spiraling number
of cases and widespread protests (termed the White Paper Revolution),
China finally abandoned its zero Covid policy (Mao, 2022). The two waves
of these surveys aim to compare attitudes before and after this
widespread and highly visible change in digital monitoring strategies.

The demographics of the 2021 and 2023 surveys are presented in
Table~\ref{tbl-demographics}.

\hypertarget{tbl-demographics}{}
\begin{table}
\caption{\label{tbl-demographics}Survey demographics }\tabularnewline

\centering
\begin{tabular}[t]{lrr}
\toprule
  & 2021 & 2023\\
\midrule
 & \num{1500.00} & \num{2007.00}\\
\bottomrule
\end{tabular}
\end{table}

As is typical of online surveys in China, the sample respondents skew
somewhat younger and more educated. Comparing the two waves, there are
some modest demographic differences (notably education and marriage)
differences between the two samples. As will be shown in
Section~\ref{sec-analysis}, these minor differences do not appear to
change any of the substantive results. Focusing on the 2023 survey, the
modal respondent is someone from a small city, male, married, working in
a white collar job at a small enterprise, who earns about 10,000 RMB a
month and has an urban \emph{hukou}. This demographic profile already
suggests that while the large-scale lockdowns that occurred in a few of
the big cities generated a lot of press, they may not be the modal or
average citizen's experience with zero Covid policies.

Taking a wide-angle view on the government's performance,
Table~\ref{tbl-gp.summary.data} compares some different measures of
government performance. While the higher level of government trust in
the central government is unsurprising (consistent with previous
literature, see CITATION), the magnitude of the gap is somewhat smaller
than in previous studies. There has been a small but statistically
significant decrease in trust of both since 2021. Most interestingly,
though, is that there was no decrease in how residents feel the
government handled their privacy information. This seems to indicate
that it not necessarily government monitoring that made residents
unhappy but instead other zero Covid policy failures.

\hypertarget{tbl-gp.q.text}{}
\begin{table}
\caption{\label{tbl-gp.q.text}Government performance questions }\tabularnewline

\centering
\resizebox{\linewidth}{!}{
\begin{tabular}[t]{l|l}
\hline
Q1 & Overall, I’m happy with the performance of the central government\\
\hline
\cellcolor{gray!6}{Q2} & \cellcolor{gray!6}{Overall, I’m happy with the performance of my local government}\\
\hline
Q3 & \makecell[c]{The government does a good job balancing the rights of citizens to be free of surveillance\\in their daily lives with the need to preserve order and prevent crime.}\\
\hline
\cellcolor{gray!6}{Q4} & \cellcolor{gray!6}{Government performance index}\\
\hline
\end{tabular}}
\end{table}

\hypertarget{tbl-gp.summary.data}{}
\begin{table}
\caption{\label{tbl-gp.summary.data}Government performance data }\tabularnewline

\centering
\resizebox{\linewidth}{!}{
\begin{tabular}[t]{lrrrrrr}
\toprule
\multicolumn{1}{c}{ } & \multicolumn{2}{c}{2021 (N=1500)} & \multicolumn{2}{c}{2023 (N=2007)} & \multicolumn{2}{c}{ } \\
\cmidrule(l{3pt}r{3pt}){2-3} \cmidrule(l{3pt}r{3pt}){4-5}
  & Mean & Std. Dev. & Mean & Std. Dev. & Diff. in Means & p\\
\midrule
\cellcolor{gray!6}{Central government performance} & \cellcolor{gray!6}{5.98} & \cellcolor{gray!6}{1.14} & \cellcolor{gray!6}{5.73} & \cellcolor{gray!6}{1.36} & \cellcolor{gray!6}{-0.25} & \cellcolor{gray!6}{0.00}\\
Local government performance & 5.55 & 1.25 & 5.35 & 1.43 & -0.20 & 0.00\\
\cellcolor{gray!6}{Government performance Q3} & \cellcolor{gray!6}{5.45} & \cellcolor{gray!6}{1.22} & \cellcolor{gray!6}{5.42} & \cellcolor{gray!6}{1.41} & \cellcolor{gray!6}{-0.03} & \cellcolor{gray!6}{0.44}\\
Government performance index & 0.78 & 0.17 & 0.75 & 0.21 & -0.03 & 0.00\\
\bottomrule
\end{tabular}}
\end{table}

Turning to the issue of specific attitudes regarding monitoring,
Table~\ref{tbl-gm.summary.data} suggests that while the differences are
not large, respondents in the second wave were more accepting of almost
all forms of monitoring. Given the phrasing of the question - ``there
are good reasons for the government to monitor you'', it seems likely
that respondents were accepting the government's framing that such
digital monitoring and control was a necessary part of the pandemic
response. Not surprisingly, and consistent with previous research (Chen,
2017; Chen and MacDonald, 2020; Li, 2016), respondents trust the
government at significantly higher levels than private corporations.
While the pandemic-era monitoring was in fact a public-private
partnership\footnote{Alibaba and Tencent served as the interface for the
  health code system while the data was analyzed and manipulated by
  local governments (McMorrow and Leng, 2022).}, respondents seem to
have a very clear delineation of which parties may acceptably gather
their data and which should not.

This result is one of the largest differences between 2021 and 2023
among all survey questions. In 2023, respondents felt that the central
government had a significantly stronger reason to monitor them compared
to 2021. Respondents also agreed that the local government had a better
case to monitor them compared to 2021, though the magnitude of the
change was not as dramatic. Other types of monitoring (private
monitoring, biometric monitoring) also exhibited a statistically
significant change in the direction of being more accepting of tracking.
The most direct interpretation of this response pattern is that
respondents fundamentally accepted the government's position that
monitoring was necessary and justified during the pandemic. Contrary to
the Western popular press reports of the White Paper Revolution, it does
not seem that most Chinese citizens were unhappy about Covid electronic
controls. The slight decrease in government trust could indicate
dissatisfaction with other Covid policies (including forced
quarantines), but it does not seem that app-based monitoring itsellf
caused any great concern among the general public.

\hypertarget{tbl-gm.q.text}{}
\begin{table}
\caption{\label{tbl-gm.q.text}Government and private monitoring questions }\tabularnewline

\centering
\resizebox{\linewidth}{!}{
\begin{tabular}[t]{l|l}
\hline
Q1 & There are good reasons for the central government to monitor the activity of users online\\
\hline
\cellcolor{gray!6}{Q2} & \cellcolor{gray!6}{There are good reasons for the local government to monitor the activity of users online}\\
\hline
Q3 & There are good reasons for private companies to monitor the activity of users online\\
\hline
\cellcolor{gray!6}{Q4} & \cellcolor{gray!6}{\makecell[c]{It doesn’t bother me to provide the government with biometric information including\\my fingerprints and face details for the purposes of monitoring public places}}\\
\hline
Q5 & \makecell[c]{It doesn’t bother me to provide private companies with biometric information including\\my fingerprints and face details for the purposes of monitoring public places}\\
\hline
\cellcolor{gray!6}{Q6} & \cellcolor{gray!6}{Government monitoring index of Q1 + Q2 + Q4}\\
\hline
Q7 & Private monitoring index of Q3 + Q5\\
\hline
\cellcolor{gray!6}{Q8} & \cellcolor{gray!6}{Total monitoring index of Q1-Q5}\\
\hline
\end{tabular}}
\end{table}

\hypertarget{tbl-gm.summary.data}{}
\begin{table}
\caption{\label{tbl-gm.summary.data}Government and private monitoring data }\tabularnewline

\centering
\resizebox{\linewidth}{!}{
\begin{tabular}[t]{lrrrrrr}
\toprule
\multicolumn{1}{c}{ } & \multicolumn{2}{c}{2021 (N=1500)} & \multicolumn{2}{c}{2023 (N=2007)} & \multicolumn{2}{c}{ } \\
\cmidrule(l{3pt}r{3pt}){2-3} \cmidrule(l{3pt}r{3pt}){4-5}
  & Mean & Std. Dev. & Mean & Std. Dev. & Diff. in Means & p\\
\midrule
\cellcolor{gray!6}{Central government monitoring} & \cellcolor{gray!6}{4.67} & \cellcolor{gray!6}{1.46} & \cellcolor{gray!6}{5.25} & \cellcolor{gray!6}{1.39} & \cellcolor{gray!6}{0.59} & \cellcolor{gray!6}{0.00}\\
Local government monitoring & 4.66 & 1.37 & 5.04 & 1.46 & 0.38 & 0.00\\
\cellcolor{gray!6}{Private company monitoring} & \cellcolor{gray!6}{2.91} & \cellcolor{gray!6}{1.56} & \cellcolor{gray!6}{3.08} & \cellcolor{gray!6}{1.84} & \cellcolor{gray!6}{0.18} & \cellcolor{gray!6}{0.00}\\
Government biometric monitoring & 4.80 & 1.50 & 5.00 & 1.59 & 0.21 & 0.00\\
\cellcolor{gray!6}{Private biometric monitoring} & \cellcolor{gray!6}{2.87} & \cellcolor{gray!6}{1.60} & \cellcolor{gray!6}{2.88} & \cellcolor{gray!6}{1.85} & \cellcolor{gray!6}{0.01} & \cellcolor{gray!6}{0.88}\\
Government monitoring index & 0.62 & 0.21 & 0.68 & 0.22 & 0.07 & 0.00\\
\cellcolor{gray!6}{Private monitoring index} & \cellcolor{gray!6}{0.31} & \cellcolor{gray!6}{0.23} & \cellcolor{gray!6}{0.33} & \cellcolor{gray!6}{0.29} & \cellcolor{gray!6}{0.02} & \cellcolor{gray!6}{0.08}\\
Total monitoring index & 0.50 & 0.17 & 0.54 & 0.19 & 0.05 & 0.00\\
\bottomrule
\end{tabular}}
\end{table}

One possible reason why respondents may believe that private
corporations are less trustworthy could arise from feeling that their
monitoring is more invasive. However, this turns out not to be the case
- respondents feel that all three entities are roughly equally likely to
monitor them. A likely interpretation of this result is that respondents
are unable to precisely identify who is monitoring them and when. When
the question about privacy is rephrased to further emphasize that these
different groups could access their private information, trust with
government sources decreases modestly compared to the previous phrasing
in Table~\ref{tbl-gm.q.text}. However, this decrease is matched by a
similar decrease in comfort with private companies monitoring them.
These results reinforce the results in Table~\ref{tbl-gm.summary.data}
and further strengthen the finding of

What may explain the lack of increased concern about digital privacy is
that respondents generally did not notice a major change in the level of
online monitoring. While the increase was statistically significant, it
was just barely at the edge of significance and amounts to less than
1/10th a standard deviation increase in perceived monitoring. Given the
invasiveness of the technological means of control employed to control
Covid, this result is surprising. Two reasonable explanations for this
divergence are 1) respondents do not consider the Covid controls to be
online monitoring and/or, in my view, more likely, 2) respondents have
already readjusted their frame of reference and no longer bring to mind
the Covid-19 era when answering this question. While the survey results
cannot arbitrate between these two explanations, both of these
explanations suggest that even a massive and intrusive increase in
surveillance has shifted attitudes about monitoring overall. If the
first explanation is true, it suggests that the kinds of monitoring that
Chinese citizens are worried about are drastically different than the
kinds Western privacy advocates are concerned about. If a government is
monitoring your every move and such activity is not considered to be
tracking your activity, then it suggests a very different set of ideas
about what is a concerning type of monitoring. If the second explanation
is true, it does suggest that respondents that view somewhat
time-limited surveillance and surveillance for a specific purpose as
being acceptable. One can easily imagine, however, such tools being used
again for periodic incidents of unrest and these results suggest that
respondents may view controls, as long as viewed as `necessary' may be
seen as acceptable at least after the fact.

\hypertarget{tbl-track.q.text}{}
\begin{table}
\caption{\label{tbl-track.q.text}Attitudes regarding tracking questions }\tabularnewline

\centering
\resizebox{\linewidth}{!}{
\begin{tabular}[t]{l|l}
\hline
Q1 & How closely do you think the central government tracks your online activity?\\
\hline
\cellcolor{gray!6}{Q2} & \cellcolor{gray!6}{How closely do you think the local government tracks your online activity?}\\
\hline
Q3 & How closely do you think private companies track your online activity?\\
\hline
\cellcolor{gray!6}{Q4} & \cellcolor{gray!6}{\makecell[c]{How comfortable are you with the central government knowing personal details about\\your activity online?}}\\
\hline
Q5 & \makecell[c]{How comfortable are you with the local government knowing personal details about\\your activity \vphantom{1} online?}\\
\hline
\cellcolor{gray!6}{Q5} & \cellcolor{gray!6}{\makecell[c]{How comfortable are you with the local government knowing personal details about\\your activity online?}}\\
\hline
\end{tabular}}
\end{table}

\hypertarget{tbl-track.summary.data}{}
\begin{table}
\caption{\label{tbl-track.summary.data}Attitudes regarding tracking summary data }\tabularnewline

\centering
\resizebox{\linewidth}{!}{
\begin{tabular}[t]{lrrrrrr}
\toprule
\multicolumn{1}{c}{ } & \multicolumn{2}{c}{2021 (N=1500)} & \multicolumn{2}{c}{2023 (N=2007)} & \multicolumn{2}{c}{ } \\
\cmidrule(l{3pt}r{3pt}){2-3} \cmidrule(l{3pt}r{3pt}){4-5}
  & Mean & Std. Dev. & Mean & Std. Dev. & Diff. in Means & p\\
\midrule
\cellcolor{gray!6}{Central government tracking - prevalence} & \cellcolor{gray!6}{4.31} & \cellcolor{gray!6}{1.30} & \cellcolor{gray!6}{4.40} & \cellcolor{gray!6}{1.44} & \cellcolor{gray!6}{0.09} & \cellcolor{gray!6}{0.05}\\
Local government tracking - prevalence & 4.22 & 1.28 & 4.35 & 1.45 & 0.13 & 0.01\\
\cellcolor{gray!6}{Private company tracking - prevalence} & \cellcolor{gray!6}{4.37} & \cellcolor{gray!6}{1.56} & \cellcolor{gray!6}{4.32} & \cellcolor{gray!6}{1.71} & \cellcolor{gray!6}{-0.05} & \cellcolor{gray!6}{0.34}\\
Central government tracking - comfort & 4.22 & 1.48 & 4.13 & 1.61 & -0.09 & 0.10\\
\cellcolor{gray!6}{Local govenrment tracking - comfort} & \cellcolor{gray!6}{4.09} & \cellcolor{gray!6}{1.49} & \cellcolor{gray!6}{4.05} & \cellcolor{gray!6}{1.63} & \cellcolor{gray!6}{-0.04} & \cellcolor{gray!6}{0.44}\\
Private company tracking - comfort & 2.64 & 1.68 & 2.54 & 1.78 & -0.11 & 0.07\\
\bottomrule
\end{tabular}}
\end{table}

Rounding out the final section of the regular survey questions are a set
of questions designed to further parse attitudes about online
monitoring, the results of which are shown in
Table~\ref{tbl-misc.summary.data}. The results of these questions
generally confirm and support the findings of all the previous question
blocks. As with most of the survey questions, there appeared to be only
a very modest change in response patterns between 2021 and 2023.
Respondents generally agree that they do not notice government tracking.
They strongly agree that the government protects their data better than
private corporations. They strongly disagree that they are willing to
give up their privacy simply to use apps for free. Finally, many
respondents feel worried about having their payment data stolen. In most
cases, it is unwise to place too much emphasis on any one question,
given respondents can misinterpret or gloss over any specific question.
In this survey, however, respondents have repeatedly indicated that they
trust the government at significantly higher rates than private
corporations and that most of users worries about being online are
related to protection of their information from corporations.

\hypertarget{tbl-misc.q.text}{}
\begin{table}
\caption{\label{tbl-misc.q.text}Attitudes on general questions }\tabularnewline

\centering
\begin{tabular}[t]{l|l}
\hline
Q1 & \makecell[c]{I don’t notice government use of technology to monitor my behavior in my \\daily life.}\\
\hline
\cellcolor{gray!6}{Q2} & \cellcolor{gray!6}{\makecell[c]{The government is likely to securely store my online personal data and \\information better than private companies.}}\\
\hline
Q3 & \makecell[c]{It doesn’t bother me if private companies sell my user data to third parties\\if it will allow me to use their applications for free.}\\
\hline
\cellcolor{gray!6}{Q4} & \cellcolor{gray!6}{I’m worried that my payment information might be stolen or compromised.}\\
\hline
\end{tabular}
\end{table}

\hypertarget{tbl-misc.summary.data}{}
\begin{table}
\caption{\label{tbl-misc.summary.data}General questions data }\tabularnewline

\centering
\resizebox{\linewidth}{!}{
\begin{tabular}[t]{lrrrrrr}
\toprule
\multicolumn{1}{c}{ } & \multicolumn{2}{c}{2021 (N=1500)} & \multicolumn{2}{c}{2023 (N=2007)} & \multicolumn{2}{c}{ } \\
\cmidrule(l{3pt}r{3pt}){2-3} \cmidrule(l{3pt}r{3pt}){4-5}
  & Mean & Std. Dev. & Mean & Std. Dev. & Diff. in Means & p\\
\midrule
\cellcolor{gray!6}{Do not notice government tracking} & \cellcolor{gray!6}{4.27} & \cellcolor{gray!6}{1.34} & \cellcolor{gray!6}{4.46} & \cellcolor{gray!6}{1.51} & \cellcolor{gray!6}{0.20} & \cellcolor{gray!6}{0.00}\\
Government secures data better than private & 5.50 & 1.27 & 5.48 & 1.36 & -0.02 & 0.71\\
\cellcolor{gray!6}{OK if apps sell my data so can use for free} & \cellcolor{gray!6}{2.27} & \cellcolor{gray!6}{1.55} & \cellcolor{gray!6}{2.44} & \cellcolor{gray!6}{1.77} & \cellcolor{gray!6}{0.17} & \cellcolor{gray!6}{0.00}\\
Payment data stolen worries & 5.64 & 1.27 & 5.50 & 1.53 & -0.14 & 0.00\\
\bottomrule
\end{tabular}}
\end{table}

One obvious objection to the finding that respondents have a higher
degree of concern with private monitoring compared to government
monitoring is that respondents are engaging in preference falsification
- they may be worried about, either consciously or subconsciously,
marking the government negatively in a survey. To address this concern,
the end of the survey employed a list experiment to measure variation in
levels of trust. The list experiment question gives respondents a list
of organizations that they may trust and then asks them to report the
number of organizations that they trust. Half of the respondents were
given a list of organizations that included a sensitive organization
(such as the central government). The other half was given a list
without the sensitive organization included. The idea is that
respondents may be more comfortable reporting that they do not trust an
organization when they do not have to consciously mark on a survey that
they do not trust it but instead is part of a mental math calculation
along with other items (Blair and Imai, 2012). List experiments have
been used across many fields to study sensitive topics such as racism,
abortion, and sexual violence (Moseson et al., 2017; Redlawsk et al.,
2010; Traunmüller et al., 2019). Since trust in corporations does not
seem likely to generate preference falsification problems, they were not
included as a separate list experiment. However, the contents of the
list items are largely technology companies so some inference can be
drawn about trust in technology companies versus the government.

The results of the list experiment are shown in
Table~\ref{tbl-listexp.summary.data}. For respondents shown the
sensitive list item, one can estimate that about 60\% of people selected
it (given that the baseline level is about 0.6 number of items selected
lower than compared to when respondents are shown the sensitive list
item). While it is hard to directly compare with the Likert-scaled
questions, note that the average on the Likert scale questions for
various trust measure of government use of data was about 4.5 out of 7,
or roughly the 65th percentile of the scale. By way of contrast, 3 out
of the 4 list items were private technology firms and the other list
item was their family. Considering private corporations, if one
speculatively assumes is that most people will select the trust in their
family list item, roughly indicating that a little over 1 out of 3 of
the private corporations on the list were mentally chosen. This roughly
accords with the average responses to trust in private corporations of
2.5 out of 7 on a Likert scale. These results are not meant to
definitively confirm that there are no issues of preference
falsification. That being said, the results do strongly parallel to the
directly asked questions, adding confidence to the interpretation of the
results of the previous tables.

\begin{table}

\end{table}

\begin{table}

\caption{\label{tbl-listexp.summary.data}List experiment summary
data}\begin{minipage}[t]{\linewidth}
\subcaption{\label{tbl-listexp.summary.data-1}Central government list experiment}

{\centering 

\centering
\begin{tabular}[t]{lrrrr}
\toprule
\multicolumn{1}{c}{ } & \multicolumn{2}{c}{2021} & \multicolumn{2}{c}{2023} \\
\cmidrule(l{3pt}r{3pt}){2-3} \cmidrule(l{3pt}r{3pt}){4-5}
  & SI not shown & SI shown & SI not shown & SI shown\\
\midrule
\cellcolor{gray!6}{Number of items selected} & \cellcolor{gray!6}{\num{2.19}} & \cellcolor{gray!6}{\num{2.84}} & \cellcolor{gray!6}{\num{2.18}} & \cellcolor{gray!6}{\num{2.84}}\\
\bottomrule
\multicolumn{5}{l}{\rule{0pt}{1em}SI = sensitive item}\\
\end{tabular}

}

\end{minipage}%
\newline
\begin{minipage}[t]{\linewidth}
\subcaption{\label{tbl-listexp.summary.data-2}Local government list experiment}

{\centering 

\centering
\begin{tabular}[t]{lrrrr}
\toprule
\multicolumn{1}{c}{ } & \multicolumn{2}{c}{2021} & \multicolumn{2}{c}{2023} \\
\cmidrule(l{3pt}r{3pt}){2-3} \cmidrule(l{3pt}r{3pt}){4-5}
  & SI not shown & SI shown & SI not shown & SI shown\\
\midrule
\cellcolor{gray!6}{Number of items selected} & \cellcolor{gray!6}{\num{2.25}} & \cellcolor{gray!6}{\num{2.86}} & \cellcolor{gray!6}{\num{2.24}} & \cellcolor{gray!6}{\num{2.77}}\\
\bottomrule
\multicolumn{5}{l}{\rule{0pt}{1em}SI = sensitive item}\\
\end{tabular}

}

\end{minipage}%

\end{table}

\hypertarget{sec-analysis}{%
\subsection{Regression Analysis}\label{sec-analysis}}

\hypertarget{conclusion}{%
\subsection{Conclusion}\label{conclusion}}

\newpage{}

\hypertarget{references}{%
\subsection{References}\label{references}}

\hypertarget{refs}{}
\begin{CSLReferences}{1}{0}
\leavevmode\vadjust pre{\hypertarget{ref-wuhanlo2021}{}}%
\emph{BBC News} (2021)
\href{https://www.bbc.com/news/world-asia-china-55628488}{Wuhan
lockdown: A year of china's fight against the covid pandemic}. Epub
ahead of print 22 January 2021.

\leavevmode\vadjust pre{\hypertarget{ref-blair2012}{}}%
Blair G and Imai K (2012)
\href{https://doi.org/10.1093/pan/mpr048}{Statistical Analysis of List
Experiments}. \emph{Political Analysis} 20(1): 47--77.

\leavevmode\vadjust pre{\hypertarget{ref-chen2017}{}}%
Chen D (2017) \href{https://doi.org/10.1177/1065912917691360}{Local
Distrust and Regime Support: Sources and Effects of Political Trust in
China}. \emph{Political Research Quarterly} 70(2): 314--326.

\leavevmode\vadjust pre{\hypertarget{ref-chen2020}{}}%
Chen D and MacDonald AW (2020)
\href{https://doi.org/10.1017/XPS.2019.15}{Bread and Circuses: Sports
and Public Opinion in China}. \emph{Journal of Experimental Political
Science} 7(1): 41--55.

\leavevmode\vadjust pre{\hypertarget{ref-li2016}{}}%
Li L (2016) \href{https://doi.org/10.1017/S0305741015001629}{Reassessing
Trust in the Central Government: Evidence from Five National Surveys}.
\emph{The China Quarterly} 225: 100--121.

\leavevmode\vadjust pre{\hypertarget{ref-mao2022}{}}%
Mao (2022)
\href{https://www.bbc.com/news/world-asia-china-63855508}{China abandons
key parts of zero-covid strategy after protests}. \emph{BBC News}. Epub
ahead of print 7 December 2022.

\leavevmode\vadjust pre{\hypertarget{ref-mcmorrow2022}{}}%
McMorrow R and Leng C (2022)
\href{https://www.ft.com/content/dee6bcc6-3fc5-4edc-814d-46dc73e67c7e}{{`}Digital
handcuffs{'}: China{'}s covid health apps govern life but are ripe for
abuse}. \emph{Financial Times}. Epub ahead of print 28 June 2022.

\leavevmode\vadjust pre{\hypertarget{ref-moseson2017}{}}%
Moseson H, Treleaven E, Gerdts C, et al. (2017)
\href{https://doi.org/10.1111/sifp.12042}{The List Experiment for
Measuring Abortion: What We Know and What We Need}. \emph{Studies in
Family Planning} 48(4): 397--405.

\leavevmode\vadjust pre{\hypertarget{ref-redlawsk2010}{}}%
Redlawsk DP, Tolbert CJ and Franko W (2010)
\href{https://doi.org/10.1177/1065912910373554}{Voters, Emotions, and
Race in 2008: Obama as the First Black President}. \emph{Political
Research Quarterly} 63(4): 875--889.

\leavevmode\vadjust pre{\hypertarget{ref-traunmuxfcller2019}{}}%
Traunmüller R, Kijewski S and Freitag M (2019)
\href{https://doi.org/10.1177/0022002719828053}{The Silent Victims of
Sexual Violence during War: Evidence from a List Experiment in Sri
Lanka}. \emph{Journal of Conflict Resolution} 63(9): 2015--2042.

\end{CSLReferences}



\end{document}
