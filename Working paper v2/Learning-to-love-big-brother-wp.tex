% Options for packages loaded elsewhere
\PassOptionsToPackage{unicode}{hyperref}
\PassOptionsToPackage{hyphens}{url}
\PassOptionsToPackage{dvipsnames,svgnames,x11names}{xcolor}
%
\documentclass[
  letterpaper,
  DIV=11,
  numbers=noendperiod]{scrartcl}

\usepackage{amsmath,amssymb}
\usepackage{iftex}
\ifPDFTeX
  \usepackage[T1]{fontenc}
  \usepackage[utf8]{inputenc}
  \usepackage{textcomp} % provide euro and other symbols
\else % if luatex or xetex
  \usepackage{unicode-math}
  \defaultfontfeatures{Scale=MatchLowercase}
  \defaultfontfeatures[\rmfamily]{Ligatures=TeX,Scale=1}
\fi
\usepackage{lmodern}
\ifPDFTeX\else  
    % xetex/luatex font selection
\fi
% Use upquote if available, for straight quotes in verbatim environments
\IfFileExists{upquote.sty}{\usepackage{upquote}}{}
\IfFileExists{microtype.sty}{% use microtype if available
  \usepackage[]{microtype}
  \UseMicrotypeSet[protrusion]{basicmath} % disable protrusion for tt fonts
}{}
\makeatletter
\@ifundefined{KOMAClassName}{% if non-KOMA class
  \IfFileExists{parskip.sty}{%
    \usepackage{parskip}
  }{% else
    \setlength{\parindent}{0pt}
    \setlength{\parskip}{6pt plus 2pt minus 1pt}}
}{% if KOMA class
  \KOMAoptions{parskip=half}}
\makeatother
\usepackage{xcolor}
\setlength{\emergencystretch}{3em} % prevent overfull lines
\setcounter{secnumdepth}{5}
% Make \paragraph and \subparagraph free-standing
\ifx\paragraph\undefined\else
  \let\oldparagraph\paragraph
  \renewcommand{\paragraph}[1]{\oldparagraph{#1}\mbox{}}
\fi
\ifx\subparagraph\undefined\else
  \let\oldsubparagraph\subparagraph
  \renewcommand{\subparagraph}[1]{\oldsubparagraph{#1}\mbox{}}
\fi


\providecommand{\tightlist}{%
  \setlength{\itemsep}{0pt}\setlength{\parskip}{0pt}}\usepackage{longtable,booktabs,array}
\usepackage{calc} % for calculating minipage widths
% Correct order of tables after \paragraph or \subparagraph
\usepackage{etoolbox}
\makeatletter
\patchcmd\longtable{\par}{\if@noskipsec\mbox{}\fi\par}{}{}
\makeatother
% Allow footnotes in longtable head/foot
\IfFileExists{footnotehyper.sty}{\usepackage{footnotehyper}}{\usepackage{footnote}}
\makesavenoteenv{longtable}
\usepackage{graphicx}
\makeatletter
\def\maxwidth{\ifdim\Gin@nat@width>\linewidth\linewidth\else\Gin@nat@width\fi}
\def\maxheight{\ifdim\Gin@nat@height>\textheight\textheight\else\Gin@nat@height\fi}
\makeatother
% Scale images if necessary, so that they will not overflow the page
% margins by default, and it is still possible to overwrite the defaults
% using explicit options in \includegraphics[width, height, ...]{}
\setkeys{Gin}{width=\maxwidth,height=\maxheight,keepaspectratio}
% Set default figure placement to htbp
\makeatletter
\def\fps@figure{htbp}
\makeatother
% definitions for citeproc citations
\NewDocumentCommand\citeproctext{}{}
\NewDocumentCommand\citeproc{mm}{%
  \begingroup\def\citeproctext{#2}\cite{#1}\endgroup}
\makeatletter
 % allow citations to break across lines
 \let\@cite@ofmt\@firstofone
 % avoid brackets around text for \cite:
 \def\@biblabel#1{}
 \def\@cite#1#2{{#1\if@tempswa , #2\fi}}
\makeatother
\newlength{\cslhangindent}
\setlength{\cslhangindent}{1.5em}
\newlength{\csllabelwidth}
\setlength{\csllabelwidth}{3em}
\newenvironment{CSLReferences}[2] % #1 hanging-indent, #2 entry-spacing
 {\begin{list}{}{%
  \setlength{\itemindent}{0pt}
  \setlength{\leftmargin}{0pt}
  \setlength{\parsep}{0pt}
  % turn on hanging indent if param 1 is 1
  \ifodd #1
   \setlength{\leftmargin}{\cslhangindent}
   \setlength{\itemindent}{-1\cslhangindent}
  \fi
  % set entry spacing
  \setlength{\itemsep}{#2\baselineskip}}}
 {\end{list}}
\usepackage{calc}
\newcommand{\CSLBlock}[1]{\hfill\break\parbox[t]{\linewidth}{\strut\ignorespaces#1\strut}}
\newcommand{\CSLLeftMargin}[1]{\parbox[t]{\csllabelwidth}{\strut#1\strut}}
\newcommand{\CSLRightInline}[1]{\parbox[t]{\linewidth - \csllabelwidth}{\strut#1\strut}}
\newcommand{\CSLIndent}[1]{\hspace{\cslhangindent}#1}

\usepackage{booktabs}
\usepackage{longtable}
\usepackage{array}
\usepackage{multirow}
\usepackage{wrapfig}
\usepackage{float}
\usepackage{colortbl}
\usepackage{pdflscape}
\usepackage{tabu}
\usepackage{threeparttable}
\usepackage{threeparttablex}
\usepackage[normalem]{ulem}
\usepackage{makecell}
\usepackage{xcolor}
\usepackage{siunitx}

  \newcolumntype{d}{S[
    input-open-uncertainty=,
    input-close-uncertainty=,
    parse-numbers = false,
    table-align-text-pre=false,
    table-align-text-post=false
  ]}
  
\KOMAoption{captions}{tableheading}
\makeatletter
\@ifpackageloaded{caption}{}{\usepackage{caption}}
\AtBeginDocument{%
\ifdefined\contentsname
  \renewcommand*\contentsname{Table of contents}
\else
  \newcommand\contentsname{Table of contents}
\fi
\ifdefined\listfigurename
  \renewcommand*\listfigurename{List of Figures}
\else
  \newcommand\listfigurename{List of Figures}
\fi
\ifdefined\listtablename
  \renewcommand*\listtablename{List of Tables}
\else
  \newcommand\listtablename{List of Tables}
\fi
\ifdefined\figurename
  \renewcommand*\figurename{Figure}
\else
  \newcommand\figurename{Figure}
\fi
\ifdefined\tablename
  \renewcommand*\tablename{Table}
\else
  \newcommand\tablename{Table}
\fi
}
\@ifpackageloaded{float}{}{\usepackage{float}}
\floatstyle{ruled}
\@ifundefined{c@chapter}{\newfloat{codelisting}{h}{lop}}{\newfloat{codelisting}{h}{lop}[chapter]}
\floatname{codelisting}{Listing}
\newcommand*\listoflistings{\listof{codelisting}{List of Listings}}
\makeatother
\makeatletter
\makeatother
\makeatletter
\@ifpackageloaded{caption}{}{\usepackage{caption}}
\@ifpackageloaded{subcaption}{}{\usepackage{subcaption}}
\makeatother
\ifLuaTeX
  \usepackage{selnolig}  % disable illegal ligatures
\fi
\usepackage{bookmark}

\IfFileExists{xurl.sty}{\usepackage{xurl}}{} % add URL line breaks if available
\urlstyle{same} % disable monospaced font for URLs
\hypersetup{
  pdftitle={Learning to love big brother: Chinese attitudes toward online privacy after the pandemic},
  pdfauthor={Andrew MacDonald},
  pdfkeywords={Digital Privacy, Covid-19, Authoriatrian
Regimes, Government Trust, China},
  colorlinks=true,
  linkcolor={blue},
  filecolor={Maroon},
  citecolor={Blue},
  urlcolor={Blue},
  pdfcreator={LaTeX via pandoc}}

\title{Learning to love big brother: Chinese attitudes toward online
privacy after the pandemic}
\author{Andrew MacDonald}
\date{2024-09-04}

\begin{document}
\maketitle
\begin{abstract}
test
\end{abstract}

\section{Introduction}\label{sec-introduction}

\section{Literature review}\label{sec-litreview}

\subsection{Hypotheses}\label{hypotheses}

\section{Data and summary statistics}\label{sec-datasummary}

The data for this project was collected via a commercial survey firm in
two waves, February of 2021 and March of 2023. In both the first and
second waves, Wuhan was oversampled, with residents of the city set to
be 10\% of respondents. The 2021 survey had an n=1500 and the second had
an n=2000. Questions on the two surveys were identical other than a
minor change to a question that referenced a specific date. The timing
of the two surveys came at two very different points in time of China's
Covid-19 experience. The first survey was conducted approximately seven
months after the last round of restrictions were lifted on the city of
Wuhan. China, at the time, was essentially closed to foreign travel but
otherwise had little in the way of day to day public health
restrictions. Nationwide, daily Covid cases hovered around the single
digits (\emph{BBC News}, 2021). China was at a very different point in
its journey in March of 2023. The year of 2022 saw widespread, intrusive
digital monitoring introduced. Many major cities, such as Shanghai,
Xi'an, and Shenzhen, underwent long and painful city-wide lockdown
procedures. At the end of 2023, under the weight of a spiraling number
of cases and widespread protests (termed the White Paper Revolution),
China finally abandoned its zero Covid policy (Mao, 2022). The two waves
of these surveys aim to compare attitudes before and after this
widespread and highly visible change in digital monitoring strategies.

The demographics of the 2021 and 2023 surveys are presented in
Table~\ref{tbl-demographics}.

\begin{longtable}[t]{llrr}

\caption{\label{tbl-demographics}Select key demographic variables}

\tabularnewline

\toprule
  &    & Mean & Std. Dev.\\
\midrule
Age &  & 33.2 & 11.6\\
\midrule
 &  & N & Pct.\\
Location & Countryside/village & 477 & 13.6\\
 & Small city & 1059 & 30.2\\
 & Mid-sized city & 840 & 24.0\\
 & Big city & 1131 & 32.2\\
Education & No formal education & 22 & 0.6\\
 & Primary & 134 & 3.8\\
 & Middle school & 384 & 10.9\\
 & High school & 843 & 24.0\\
 & University & 1914 & 54.6\\
 & Advanced studies/Graduate school & 210 & 6.0\\
Gender & Female & 1711 & 48.8\\
 & Male & 1796 & 51.2\\
Marriage status & Single & 1101 & 31.4\\
 & In a relationship & 569 & 16.2\\
 & Married & 1744 & 49.7\\
 & Divorced & 93 & 2.7\\
Party member status & Yes & 483 & 13.8\\
 & No & 3024 & 86.2\\
Communist Youth League status & Yes & 1116 & 31.8\\
 & No & 2391 & 68.2\\
Income & 0-2,999 & 275 & 7.8\\
 & 3,000-5,999 & 822 & 23.4\\
 & 6,000-9,999 & 899 & 25.6\\
 & 10,000-19,999 & 962 & 27.4\\
 & 20,000-49,999 & 385 & 11.0\\
 & 50,000-99,999 & 94 & 2.7\\
 & More than 100,000 & 70 & 2.0\\
Year & 2021 & 1500 & 42.8\\
 & 2023 & 2007 & 57.2\\
\bottomrule

\end{longtable}

As is typical of online surveys in China, the sample respondents skew
somewhat younger and more educated. Comparing the two waves, there are
some modest demographic differences (notably education and marriage)
differences between the two samples. As will be shown in
Section~\ref{sec-analysis}, these minor differences do not appear to
change any of the substantive results. Focusing on the 2023 survey, the
modal respondent is someone from a small city, male, married, working in
a white collar job at a small enterprise, who earns about 10,000 RMB a
month and has an urban \emph{hukou}.

To simplify the analysis that follows, the variables are recoded such
that income is divided into three categories (low, middle, and high) and
education is divided into two categories, those with college education
and those without. The regressions in the following sections were tested
with alternate specifications of these categorical variables (code for
these regressions available at the author's webite) and the key results
were unaffected.

\begin{table}

\caption{\label{tbl-respvarindex}Questions asking about attitudes toward
government monitoring}

\centering{

\centering
\begin{tabular}[t]{l|>{\raggedright\arraybackslash}p{5in}}
\hline
GM1 & There are good reasons for the central government to monitor the activity of users online\\
\hline
\cellcolor{gray!6}{GM2} & \cellcolor{gray!6}{There are good reasons for the local government to monitor the activity of users online}\\
\hline
TRACK1 & How comfortable are you with the central government knowing personal details about your activity online?\\
\hline
\cellcolor{gray!6}{TRACK2} & \cellcolor{gray!6}{How comfortable are you with the local government knowing personal details about your activity online?}\\
\hline
\end{tabular}

}

\end{table}%

The key response variable for the following analysis is an index
variable created by combining the results of the four questions in
Table~\ref{tbl-respvarindex}, rescaled to be between zero and one. It is
true that previous research has found that Chinese respondents place
lower levels of trust in local governments as compared to central
governments (CITE). However, the correlation between \texttt{GM1} and
\texttt{GM2} is 0.84 and the correlation between \texttt{TRACK1} and
\texttt{TRACK2} is 0.87. Additionally, as shown in the online
appendices, each of the major regression results in
Section~\ref{sec-analysis} do not demonstrate major changes if the same
variables are regressed on each of the survey question items
individually. Indicating their close relationship, the variables taken
together have a Cronbach's \(\alpha\) of 0.81. Intuitively, this high
level of relatedness makes sense as while respondents have some
background attitudes about the difference between central and local
governments, they may not easily be able to identify at which level
government tracking occurs.

Finally, there has been some debate as to the extent of preference
falsification on online surveys in China. To test for preference
falsification, the surveys also contained several list experiment
questions. List experiments have been used in a number of surveys (CITE)
to allow residents a confidential method to express their true attitudes
about topics such as racial views, sexual assault experience, and
political views. While (CITE) points out that list experiments should
not be seen as silver bullet to the problem of preference falsification,
the results of the list experiments (also available in the online
appendices) roughly match the answers to the component questions that
comprise the response variable index.

Overall, the survey data should provide a robust test to arbitrate
between the hypotheses posed in Section~\ref{sec-litreview}.

\section{Analysis}\label{sec-analysis}

\subsection{Predicting citizen attitudes about government
monitoring}\label{predicting-citizen-attitudes-about-government-monitoring}

The first set of models considers the question of which demographic
variables predict variation in attitudes toward being tracked by the
government. (stuff related to lit review)

The key predictor variables included in Table~\ref{tbl-citizenatt} are
the tech savvy index (TSI) and the knowledge index (KI). The items used
to construct each are listed in Section~\ref{sec-appendix}. The
Cronbach's \(\alpha\) for each set of variables is 0.8 and 0.55
respectively, indicating that the questions are suitable for use in an
index.

\begin{table}

\caption{\label{tbl-citizenatt}Demographic predictors of attitude toward
government privacy}

\centering{

\centering
\begin{tabular}[t]{lcccccc}
\toprule
  & (1) & (2) & (3) & (4) & (5) & (6)\\
\midrule
(Intercept) & \num{0.573}*** & \num{0.548}*** & \num{0.488}*** & \num{0.506}*** & \num{0.504}*** & \num{0.576}***\\
 & (\num{0.019}) & (\num{0.019}) & (\num{0.021}) & (\num{0.023}) & (\num{0.022}) & (\num{0.022})\\
Age & \num{0.001}*** & \num{0.001}** & \num{0.002}*** & \num{0.001}*** & \num{0.002}*** & \num{0.001}***\\
 & (\num{0.000}) & (\num{0.000}) & (\num{0.000}) & (\num{0.000}) & (\num{0.000}) & (\num{0.000})\\
College education & \num{0.010} & \num{0.014}+ & \num{0.006} & \num{-0.026} & \num{0.006} & \num{0.014}+\\
 & (\num{0.008}) & (\num{0.008}) & (\num{0.008}) & (\num{0.017}) & (\num{0.008}) & (\num{0.008})\\
Middle income & \num{0.000} & \num{0.002} & \num{-0.003} & \num{-0.004} & \num{-0.003} & \num{0.001}\\
 & (\num{0.008}) & (\num{0.008}) & (\num{0.008}) & (\num{0.008}) & (\num{0.008}) & (\num{0.008})\\
High income & \num{-0.015} & \num{-0.012} & \num{-0.024} & \num{-0.025} & \num{-0.024} & \num{-0.014}\\
 & (\num{0.017}) & (\num{0.017}) & (\num{0.017}) & (\num{0.017}) & (\num{0.017}) & (\num{0.017})\\
Male & \num{-0.004} & \num{-0.003} & \num{-0.009} & \num{-0.010} & \num{-0.040}* & \num{-0.004}\\
 & (\num{0.007}) & (\num{0.007}) & (\num{0.007}) & (\num{0.007}) & (\num{0.016}) & (\num{0.007})\\
Not a party member & \num{-0.033}** & \num{-0.034}*** & \num{-0.030}** & \num{-0.029}** & \num{-0.030}** & \num{-0.032}**\\
 & (\num{0.010}) & (\num{0.010}) & (\num{0.010}) & (\num{0.010}) & (\num{0.010}) & (\num{0.010})\\
Location: small city & \num{0.000} & \num{0.002} & \num{-0.003} & \num{-0.001} & \num{-0.003} & \num{0.002}\\
 & (\num{0.011}) & (\num{0.011}) & (\num{0.011}) & (\num{0.011}) & (\num{0.011}) & (\num{0.011})\\
Location: mid city & \num{0.007} & \num{0.009} & \num{0.000} & \num{0.001} & \num{0.000} & \num{0.008}\\
 & (\num{0.012}) & (\num{0.012}) & (\num{0.012}) & (\num{0.012}) & (\num{0.012}) & \vphantom{1} (\num{0.012})\\
Location: big city & \num{0.013} & \num{0.014} & \num{0.001} & \num{0.002} & \num{0.001} & \num{0.013}\\
 & (\num{0.012}) & (\num{0.012}) & (\num{0.012}) & (\num{0.012}) & (\num{0.012}) & (\num{0.012})\\
Year 2023 &  & \num{0.037}*** & \num{0.039}*** & \num{0.038}*** & \num{0.038}*** & \num{0.036}***\\
 &  & (\num{0.007}) & (\num{0.007}) & (\num{0.007}) & (\num{0.007}) & (\num{0.007})\\
TSI &  &  & \num{0.118}*** & \num{0.010} & \num{0.017} & \\
 &  &  & (\num{0.017}) & (\num{0.055}) & (\num{0.052}) & \\
TSI x education &  &  &  & \num{0.069}* &  & \\
 &  &  &  & (\num{0.034}) &  & \\
TSI x sex &  &  &  &  & \num{0.065}* & \\
 &  &  &  &  & (\num{0.032}) & \\
KI &  &  &  &  &  & \num{-0.058}**\\
 &  &  &  &  &  & (\num{0.022})\\
\midrule
Num.Obs. & \num{3507} & \num{3507} & \num{3507} & \num{3507} & \num{3507} & \num{3507}\\
R2 & \num{0.008} & \num{0.016} & \num{0.029} & \num{0.030} & \num{0.030} & \num{0.018}\\
F & \num{3.271} & \num{5.846} & \num{9.539} & \num{9.105} & \num{9.107} & \num{5.928}\\
\bottomrule
\multicolumn{7}{l}{\rule{0pt}{1em}+ p $<$ 0.1, * p $<$ 0.05, ** p $<$ 0.01, *** p $<$ 0.001}\\
\multicolumn{7}{l}{\rule{0pt}{1em}Reference values: no college education, low income, female, party member, countryside}\\
\multicolumn{7}{l}{\rule{0pt}{1em}Standard deviation of the response variable:  0.2}\\
\end{tabular}

}

\end{table}%

The coefficients generally indicate effects in the direction expected.
Older respondents are more accepting of government tracking, while being
male and not being a member of the party predict lower acceptance of
tracking. Curiously, living in a big city has a positive relationship
with tracking acceptance, as does the year 2023. Being tech savvy is
positively related to acceptance of tracking, while having knowledge of
privacy is negatively associated with the response variable.

However, with respect to the magnitude of the coefficients, the response
variable is scaled between zero and one with a standard deviation of
0.2. Given this scaling, the coefficients of the categorical variables
all have rather small effect sizes - being in year 2023 instead of year
2021 produces a shift in the response variable of about a third of a
standard deviation. For the other categorical variables, while some
reach significance, they have even smaller effect sizes. The tech savvy
index has a standard deviation of 0.21. The coefficient of \texttt{TSI}
indicates the impact of a one unit change in the index (going from its
minimum to its maximum) on the response variable. However, a more
typical shift in \texttt{TSI} produces an effect only one fifth as
large, or about a fifth to a tenth of a standard deviation change in the
response variable. Similarly, for \texttt{KI}, a typical shift in the
predictor variable leads to a nearly negligible change in the response
variable.

As can be inferred from the results in this table, the model fit is
relatively poor. The poor model fit can also be seen in the very low
\(R^2\) values and in the extremely poor model residuals (available in
the online appendices). Taken together, these results indicate that the
available demographic factors do a poor job explaining variation in
attitudes towards government tracking.

\subsection{Does government trust affect
attitudes?}\label{does-government-trust-affect-attitudes}

Another plausible relationship is that generalized trust in government
is a strong predictor of attitudes toward government monitoring. Similar
to the key variables in the preceding section, the measure of both local
government and central government performance are combined into an index
(\texttt{GPI}). As noted in Section~\ref{sec-datasummary}, while there
has been an observed gap in measurement of the two concepts in previous
literature (and in these surveys), the responses to the two questions
are nevertheless highly correlated (\(\rho=0.73\)). To the extent that
they measure disjoint opinions, models 1b and 1c take central government
performance alone and local government performance alone, as the
response variables.

\begin{table}

\caption{\label{tbl-governmentatt}}

\centering{

\captionsetup{labelsep=none}

\centering
\begin{tabular}[t]{lcccccc}
\toprule
  & (1) & (1a) & (1b) & (2) & (3) & (4)\\
\midrule
(Intercept) & \num{0.214}*** & \num{0.212}*** & \num{0.299}*** & \num{0.236}*** & \num{0.266}*** & \num{0.297}***\\
 & (\num{0.021}) & (\num{0.021}) & (\num{0.020}) & (\num{0.024}) & (\num{0.027}) & (\num{0.031})\\
Age & \num{0.001}** & \num{0.001}** & \num{0.001}*** & \num{0.001}*** & \num{0.001}** & \num{0.001}***\\
 & (\num{0.000}) & (\num{0.000}) & (\num{0.000}) & (\num{0.000}) & (\num{0.000}) & (\num{0.000})\\
College education & \num{0.012}+ & \num{0.012}+ & \num{0.013}+ & \num{-0.026} & \num{0.012}+ & \num{-0.035}\\
 & (\num{0.007}) & (\num{0.007}) & (\num{0.007}) & (\num{0.024}) & (\num{0.007}) & (\num{0.024})\\
Middle income & \num{-0.003} & \num{-0.001} & \num{-0.003} & \num{-0.003} & \num{-0.003} & \num{-0.003}\\
 & (\num{0.007}) & (\num{0.007}) & (\num{0.007}) & (\num{0.007}) & (\num{0.007}) & (\num{0.007})\\
High income & \num{-0.013} & \num{-0.010} & \num{-0.015} & \num{-0.013} & \num{-0.013} & \num{-0.013}\\
 & (\num{0.015}) & (\num{0.015}) & (\num{0.015}) & (\num{0.015}) & (\num{0.015}) & (\num{0.015})\\
Male & \num{-0.005} & \num{-0.006} & \num{-0.003} & \num{-0.005} & \num{-0.005} & \num{-0.005}\\
 & (\num{0.006}) & (\num{0.006}) & (\num{0.006}) & (\num{0.006}) & (\num{0.006}) & (\num{0.006})\\
Not a party member & \num{-0.028}** & \num{-0.026}** & \num{-0.031}*** & \num{-0.028}** & \num{-0.029}** & \num{-0.029}**\\
 & (\num{0.009}) & (\num{0.009}) & (\num{0.009}) & (\num{0.009}) & (\num{0.009}) & (\num{0.009})\\
Location: small city & \num{-0.006} & \num{-0.003} & \num{-0.007} & \num{-0.006} & \num{-0.006} & \num{-0.006}\\
 & (\num{0.010}) & (\num{0.010}) & (\num{0.010}) & (\num{0.010}) & (\num{0.010}) & (\num{0.010})\\
Location: mid city & \num{-0.001} & \num{0.008} & \num{-0.006} & \num{-0.001} & \num{-0.001} & \num{-0.001}\\
 & (\num{0.011}) & (\num{0.011}) & (\num{0.011}) & (\num{0.011}) & (\num{0.011}) & (\num{0.011})\\
Location: big city & \num{0.011} & \num{0.022}* & \num{0.002} & \num{0.010} & \num{0.010} & \num{0.010}\\
 & (\num{0.010}) & (\num{0.010}) & (\num{0.011}) & (\num{0.010}) & (\num{0.010}) & (\num{0.010})\\
Year 2023 & \num{0.052}*** & \num{0.053}*** & \num{0.047}*** & \num{0.052}*** & \num{-0.024} & \num{-0.030}\\
 & (\num{0.006}) & (\num{0.006}) & (\num{0.006}) & (\num{0.006}) & (\num{0.025}) & (\num{0.025})\\
GPI & \num{0.431}*** &  &  & \num{0.352}*** & \num{0.269}*** & \num{0.158}*\\
 & (\num{0.015}) &  &  & (\num{0.050}) & (\num{0.054}) & (\num{0.077})\\
CG performance &  & \num{0.406}*** &  &  &  & \\
 &  & (\num{0.014}) &  &  &  & \\
LG performance &  &  & \num{0.344}*** &  &  & \\
 &  &  & (\num{0.014}) &  &  & \\
GPI x education &  &  &  & \num{0.050}+ &  & \num{0.062}*\\
 &  &  &  & (\num{0.030}) &  & (\num{0.030})\\
GPI x year &  &  &  &  & \num{0.098}** & \num{0.105}***\\
 &  &  &  &  & (\num{0.031}) & (\num{0.032})\\
\midrule
Num.Obs. & \num{3507} & \num{3507} & \num{3507} & \num{3507} & \num{3507} & \num{3507}\\
R2 & \num{0.207} & \num{0.200} & \num{0.165} & \num{0.207} & \num{0.209} & \num{0.210}\\
F & \num{82.827} & \num{79.326} & \num{62.844} & \num{76.189} & \num{76.920} & \num{71.388}\\
\bottomrule
\multicolumn{7}{l}{\rule{0pt}{1em}+ p $<$ 0.1, * p $<$ 0.05, ** p $<$ 0.01, *** p $<$ 0.001}\\
\multicolumn{7}{l}{\rule{0pt}{1em}Reference values: no college education, low income, female, party member, countryside}\\
\multicolumn{7}{l}{\rule{0pt}{1em}Standard deviation of the response variable:  0.2}\\
\end{tabular}

}

\end{table}%

The coefficients on in Table~\ref{tbl-governmentatt} indicate that
government performance is a much stronger predictor of attitudes towards
government tracking than the index demographic variables. In model 1, a
one standard deviation increase in the government performance index
(0.2) predicts about a one third of a standard deviation change in
attitudes towards government tracking, a relatively significant effect
for an attitudinal survey. Additionally, the two interaction terms are
also both significant. The effect of these interaction terms can be
viewed in Figure~\ref{fig-marginplotperform}. Education lessens the
impact of government performance on acceptance of government tracking
while year 2023 increases the impact. Finally, the model fit diagnostics
have improved, indicating a better model fit.

\begin{figure}

\centering{

\centering{

\includegraphics{Learning-to-love-big-brother-wp_files/figure-pdf/fig-marginplotperform-1.pdf}

}

\subcaption{\label{fig-marginplotperform-1}GPI x education}

\centering{

\includegraphics{Learning-to-love-big-brother-wp_files/figure-pdf/fig-marginplotperform-2.pdf}

}

\subcaption{\label{fig-marginplotperform-2}GPI x year}

}

\caption{\label{fig-marginplotperform}Marginal effect plots of
interaction terms}

\end{figure}%

As expected, government performance, controlling for demographic
factors, is a relatively strong predictor of acceptance of government
performance. According to the model, a positive view of government
performance is associated with an increased acceptance of government
monitoring. However, the relationship is likely more complex than this
story, particularly given the tumultuous events of the 2022 Covid-19
lockdowns in China. To further explore how these factors interact with
each other, the next section develops a mediation model to better
understand the causal factors at play.

\subsection{Changing attitudes since the
pandemic}\label{changing-attitudes-since-the-pandemic}

The following directed acyclic graph (DAG) indicates the hypothesized
causal process that generates the observed outcome variable, tracking
acceptance. In Figure~\ref{fig-dag}, \texttt{TA} represents tracking
acceptance, \texttt{GP} represents government performance approval,
\texttt{DEMO} represents demographic characteristics, and \texttt{COVID}
represents respondents' Covid-19 experience.

\begin{figure}

\centering{

\includegraphics{Learning-to-love-big-brother-wp_files/figure-pdf/fig-dag-1.pdf}

}

\caption{\label{fig-dag}Causal process}

\end{figure}%

To operationalize this model, the first step is to create a latent
demographics variable with the demographic factors previously used in
regressions in the earlier sections loading onto this variable.
\texttt{COVID} is operationalized by the \texttt{year} variable.
Admittedly, this is not a precise mapping. The \texttt{year} variable
actually measures all changes between survey waves not accounted for by
other variables. With respect to government tracking acceptance
attitudes, however, this assumption can be justified by the fact that
the Covid-19 experience was both a daily and often traumatic one for the
Chinese public; it was a time period that involved constant and invasive
technological monitoring. If the model does indicate that the
\texttt{year} variable predicts significant change in the response
variable over the two year difference between survey waves, it would be
hard to imagine any other plausible cause. Nevertheless, it is important
to keep in mind that it is only a proxy measurement. These variables
(\texttt{year} for \texttt{COVID}, \texttt{GP}, \texttt{TA}, and
\texttt{DEMO}) are entered into a structural equation model with paths
constrained to that described by the DAG in Figure~\ref{fig-dag} and
then the parameters are estimated using the \texttt{lavaan} library in
\texttt{R}.

\begin{table}

\caption{\label{tbl-mediationmodel}Mediation model results}

\centering{

\centering
\begin{tabular}[t]{lc}
\toprule
  & (1)\\
\midrule
DEMO to GP & \num{-0.008}\\
 & \vphantom{1} (\num{0.005})\\
GP to TA & \num{0.430}***\\
 & (\num{0.015})\\
DEMO to TA & \num{-0.010}*\\
 & (\num{0.005})\\
COVID to TA & \num{0.053}***\\
 & (\num{0.006})\\
COVID to GP & \num{-0.037}***\\
 & (\num{0.007})\\
DEMO to GP to TA & \num{-0.003}\\
 & (\num{0.002})\\
COVID to GP to TA & \num{-0.002}***\\
 & (\num{0.000})\\
\midrule
Num.Obs. & \num{3507}\\
AIC & \num{52129.6}\\
BIC & \num{52246.7}\\
\bottomrule
\multicolumn{2}{l}{\rule{0pt}{1em}+ p $<$ 0.1, * p $<$ 0.05, ** p $<$ 0.01, *** p $<$ 0.001}\\
\multicolumn{2}{l}{\rule{0pt}{1em}Demographic factor variable loadings omitted}\\
\end{tabular}

}

\end{table}%

The results from this analysis in Table~\ref{tbl-mediationmodel} confirm
some of the previous findings and also reveal some interesting new
features of the data. As before, government performance plays an
important role in determining tracking acceptance. Similarly,
demographic questions do help predict tracking acceptance, but only by a
little bit. Demographics do not help predict views on government
performance, and therefore, it is not surprising that demographics does
not have an indirect effect on tracking acceptance through government
performance either. However, the Covid-19 experience, as operationalized
by the year variable, does 1) positively increase tracking acceptance
(direct path) 2) negatively decreases government performance evaluations
(direct path) and 3) negatively decreases tracking acceptance through
government performance evaluations (indirect path). The size of the
coefficients indicates that the sum of the effects is still positive on
tracking acceptance.

\subsection{Comparing attitudes to private
monitoring}\label{comparing-attitudes-to-private-monitoring}

Finally, it is interesting to compare the determinants of government
tracking attitudes with those that determine attitudes toward private
tracking of personal information. T compare these two,
Table~\ref{tbl-privatepublic} includes models that use the previous
acceptance of government tracking index (\texttt{Public\ TA}) as the
variable alongside a similarly constructed index variable that measures
acceptance of private monitoring (\texttt{Private\ TA}).

\begin{table}

\caption{\label{tbl-privatepublic}Comparison of public vs.~private
tracking acceptance}

\centering{

\centering
\begin{tabular}[t]{lcccccc}
\toprule
\multicolumn{1}{c}{ } & \multicolumn{3}{c}{Public TA} & \multicolumn{3}{c}{Private TA} \\
\cmidrule(l{3pt}r{3pt}){2-4} \cmidrule(l{3pt}r{3pt}){5-7}
  & (1a) & (1b) & (1c) & (2a) & (2b) & (2c)\\
\midrule
(Intercept) & \num{0.488}*** & \num{0.576}*** & \num{0.214}*** & \num{0.270}*** & \num{0.432}*** & \num{0.344}***\\
 & (\num{0.021}) & (\num{0.022}) & (\num{0.021}) & (\num{0.025}) & (\num{0.026}) & (\num{0.027})\\
Age & \num{0.002}*** & \num{0.001}*** & \num{0.001}** & \num{0.000} & \num{-0.001}** & \num{-0.001}**\\
 & (\num{0.000}) & (\num{0.000}) & (\num{0.000}) & (\num{0.000}) & (\num{0.000}) & (\num{0.000})\\
College education & \num{0.006} & \num{0.014}+ & \num{0.012}+ & \num{-0.027}** & \num{-0.010} & \num{-0.010}\\
 & (\num{0.008}) & (\num{0.008}) & (\num{0.007}) & (\num{0.009}) & (\num{0.009}) & \vphantom{1} (\num{0.009})\\
Middle income & \num{-0.003} & \num{0.001} & \num{-0.003} & \num{-0.059}*** & \num{-0.049}*** & \num{-0.049}***\\
 & (\num{0.008}) & (\num{0.008}) & (\num{0.007}) & (\num{0.009}) & (\num{0.009}) & (\num{0.009})\\
High income & \num{-0.024} & \num{-0.014} & \num{-0.013} & \num{-0.027} & \num{-0.004} & \num{-0.002}\\
 & (\num{0.017}) & (\num{0.017}) & (\num{0.015}) & (\num{0.019}) & (\num{0.020}) & (\num{0.020})\\
Male & \num{-0.009} & \num{-0.004} & \num{-0.005} & \num{-0.006} & \num{0.006} & \num{0.007}\\
 & (\num{0.007}) & (\num{0.007}) & (\num{0.006}) & (\num{0.008}) & (\num{0.008}) & \vphantom{1} (\num{0.008})\\
Not a party member & \num{-0.030}** & \num{-0.032}** & \num{-0.028}** & \num{-0.010} & \num{-0.016} & \num{-0.017}\\
 & (\num{0.010}) & (\num{0.010}) & (\num{0.009}) & (\num{0.012}) & (\num{0.012}) & (\num{0.012})\\
Location: small city & \num{-0.003} & \num{0.002} & \num{-0.006} & \num{-0.043}** & \num{-0.032}* & \num{-0.033}*\\
 & (\num{0.011}) & (\num{0.011}) & (\num{0.010}) & (\num{0.013}) & (\num{0.013}) & (\num{0.013})\\
Location: mid city & \num{0.000} & \num{0.008} & \num{-0.001} & \num{-0.056}*** & \num{-0.038}** & \num{-0.039}**\\
 & (\num{0.012}) & (\num{0.012}) & (\num{0.011}) & (\num{0.014}) & (\num{0.014}) & (\num{0.014})\\
Location: big city & \num{0.001} & \num{0.013} & \num{0.011} & \num{-0.051}*** & \num{-0.024}+ & \num{-0.024}+\\
 & (\num{0.012}) & (\num{0.012}) & (\num{0.010}) & (\num{0.014}) & (\num{0.014}) & (\num{0.014})\\
Year 2023 & \num{0.039}*** & \num{0.036}*** & \num{0.052}*** & \num{0.006} & \num{0.001} & \num{0.005}\\
 & (\num{0.007}) & (\num{0.007}) & (\num{0.006}) & (\num{0.008}) & (\num{0.008}) & (\num{0.008})\\
TSI & \num{0.118}*** &  &  & \num{0.252}*** &  & \\
 & (\num{0.017}) &  &  & (\num{0.021}) &  & \\
KI &  & \num{-0.058}** &  &  & \num{-0.069}** & \\
 &  & (\num{0.022}) &  &  & (\num{0.027}) & \\
GPI &  &  & \num{0.431}*** &  &  & \num{0.071}***\\
 &  &  & (\num{0.015}) &  &  & (\num{0.020})\\
\midrule
Num.Obs. & \num{3507} & \num{3507} & \num{3507} & \num{3507} & \num{3507} & \num{3507}\\
R2 & \num{0.029} & \num{0.018} & \num{0.207} & \num{0.057} & \num{0.018} & \num{0.020}\\
F & \num{9.539} & \num{5.928} & \num{82.827} & \num{19.081} & \num{5.865} & \num{6.450}\\
\bottomrule
\multicolumn{7}{l}{\rule{0pt}{1em}+ p $<$ 0.1, * p $<$ 0.05, ** p $<$ 0.01, *** p $<$ 0.001}\\
\multicolumn{7}{l}{\rule{0pt}{1em}Reference values: no college education, low income, female, party member, countryside}\\
\multicolumn{7}{l}{\rule{0pt}{1em}Standard deviation of Public TA is:  0.2}\\
\multicolumn{7}{l}{\rule{0pt}{1em}Standard deviation of Private TA is:  0.24}\\
\end{tabular}

}

\end{table}%

The results here suggest that the determinants of attitudes towards
private company tracking are somewhat different than those that
determine attitudes towards public tracking. The \texttt{year} variable
is not significant for the private tracking models, while the tech savvy
coefficient has roughly doubled. Unsurprisingly, government performance
is also not related to private tracking acceptance. Finally, the
intercept is generally lower for private tracking acceptance, indicating
that respondents are less willing to accept private tracking, all things
being equal.

\section{Conclusion}\label{sec-conclusion}

\newpage{}

\section{References}\label{references}

\phantomsection\label{refs}
\begin{CSLReferences}{1}{1}
\bibitem[\citeproctext]{ref-wuhanlo2021}
\emph{BBC News} (2021)
\href{https://www.bbc.com/news/world-asia-china-55628488}{Wuhan
lockdown: A year of china's fight against the covid pandemic}. Epub
ahead of print 22 January 2021.

\bibitem[\citeproctext]{ref-mao2022}
Mao (2022)
\href{https://www.bbc.com/news/world-asia-china-63855508}{China abandons
key parts of zero-covid strategy after protests}. \emph{BBC News}. Epub
ahead of print 7 December 2022.

\end{CSLReferences}

\newpage{}

\section{Appendix}\label{sec-appendix}

\subsection{Index variable
definitions}\label{index-variable-definitions}

\begin{table}

\caption{\label{tbl-ts.q.text}Tech savvy index questions}

\centering{

\centering
\begin{tabular}[t]{l|l}
\hline
Q1 & How would you rate your general ability to use a computer?\\
\hline
Q2 & How would you rate your skill at fixing a computer?\\
\hline
Q3 & How would you rate your ability to program a computer?\\
\hline
\end{tabular}

}

\end{table}%

\begin{table}

\caption{\label{tbl-kw.q.text}Knowledge index questions}

\centering{

\centering
\begin{tabular}[t]{l|l}
\hline
Q1 & I am very concerned about my privacy online\\
\hline
Q2 & I spend a lot of time reading about technology related privacy issues\\
\hline
Q3 & In the last year, I have had discussions with my friends about online privacy issues\\
\hline
Q4 & I feel like I know exactly how much privacy I have online\\
\hline
Q5 & Have you heard of the social credit system (official terminology)?\\
\hline
\end{tabular}

}

\end{table}%

\begin{table}

\caption{\label{tbl-gp.q.text}Government performance index questions}

\centering{

\centering
\begin{tabular}[t]{l|>{\raggedright\arraybackslash}p{5in}}
\hline
Q1 & Overall, I’m happy with the performance of the central government\\
\hline
\cellcolor{gray!6}{Q2} & \cellcolor{gray!6}{Overall, I’m happy with the performance of my local government}\\
\hline
\end{tabular}

}

\end{table}%



\end{document}
